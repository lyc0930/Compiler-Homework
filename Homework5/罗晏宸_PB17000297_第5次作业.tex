\documentclass{article}
\usepackage[UTF8]{ctex}
\usepackage[T1]{fontenc}
\usepackage[utf8]{inputenc}
\usepackage{float}
\usepackage{placeins}
\usepackage{latexsym}
\usepackage{amsmath}
\usepackage{amsthm}
\usepackage{amssymb}
\usepackage{listings}
\usepackage{xcolor}
\usepackage{tikz}
\usepackage{tabularx}
\usepackage{multirow}
\usepackage{hyperref}

\title{Homework 5}
\author{PB17000297 罗晏宸}
\date{October 15 2019}

\hypersetup{
    colorlinks=true,
    linkcolor = black,
    urlcolor=blue!30!green,
}

\begin{document}
\maketitle

\section{Exercise 4.12}
文法如下:
\begin{align*}
    S &\rightarrow (L)\ |\ a \\
    L &\rightarrow L,\,S\ |\ S
\end{align*}
\subparagraph{(a)}
写一个翻译方案,它输出每个$a$的嵌套深度。例如,对于句子$(a,\,(a,\,a))$,输出的结果是$1\quad2\quad2$。
\subparagraph{(b)}
写一个翻译方案,它打印出每个$a$在句子中是第几个字符。例如,当句子是$(a,\,(a,\,(a,\,a),(a)))$时,打印的结果是$2\quad5\quad8\quad10\quad14$。
\\
\section{Non-textbook Exercise}
\paragraph{(1)}
给出习题4.12(a)和(b)的翻译方案所对应的属性栈代码。
\\
\paragraph{解}


\paragraph{(2)}
给出习题4.12(a)和(b)的翻译方案所对应的YACC语义代码。
\\
\paragraph{解}

\paragraph{(3)}
针对以下文法
\begin{align*}
    E \rightarrow &\ E\ \texttt{'>'}\ E \\
    |&\  E\ \texttt{'<'}\ E \\
    |&\  \textbf{number}
\end{align*}
设计语法制导定义,使之能计算诸如\texttt{1 < 2 < 3}的表达式值为True;而计算表达式\texttt{1 < 5 > 3}的值也为True。
\\
\paragraph{解}

\end{document}