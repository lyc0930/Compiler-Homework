\documentclass{article}
\usepackage[UTF8]{ctex}
\usepackage[T1]{fontenc}
\usepackage[utf8]{inputenc}
\usepackage{float}
\usepackage{placeins}
\usepackage{latexsym}
\usepackage{amsmath}
\usepackage{amsthm}
\usepackage{amssymb}
\usepackage{listings}
\usepackage{xcolor}
\usepackage{tikz}
\usepackage{tabularx}
\usepackage{multirow}
\usepackage{hyperref}
\usepackage{multicol}
\usepackage{geometry}
\title{Homework 5}
\author{PB17000297 罗晏宸}
\date{October 15 2019}

\hypersetup{
    colorlinks=true,
    linkcolor = black,
    urlcolor=blue!30!green,
}

\begin{document}
\maketitle

\section{Exercise 4.12}
文法如下:
\begin{align*}
    S &\rightarrow (L)\ |\ a \\
    L &\rightarrow L,\,S\ |\ S
\end{align*}
\subparagraph{(a)}
写一个翻译方案,它输出每个$a$的嵌套深度。例如,对于句子$(a,\,(a,\,a))$,输出的结果是$1\quad2\quad2$。
\subparagraph{(b)}
写一个翻译方案,它打印出每个$a$在句子中是第几个字符。例如,当句子是$(a,\,(a,\,(a,\,a),(a)))$时,打印的结果是$2\quad5\quad8\quad10\quad14$。

\paragraph{解}
\subparagraph{(a)}
用继承属性\texttt{depth}表示嵌套深度,则翻译方案如下:
\begin{align*}
    S' \rightarrow\ & &&\texttt{\{ }S\texttt{.depth = 0; \}}\\
                     &S&& \\
    S \rightarrow\ & &&\texttt{\{ }L\texttt{.depth = }S\texttt{.depth + 1; \}}\\
                    &(L)&& \\
    S \rightarrow\  &a&& \texttt{\{ print(}S\texttt{.depth); \}}\\
    L \rightarrow\ & &&\texttt{\{ }L_1\texttt{.depth = }L\texttt{.depth; \}}\\
                    &L_1,&& \texttt{\{ }S\texttt{.depth = }L\texttt{.depth; \}}\\
                    &S&& \\
    L \rightarrow\ & &&\texttt{\{ }S\texttt{.depth = }L\texttt{.depth; \}}\\
                    &S&&
\end{align*}
\subparagraph{(b)}
用继承属性\texttt{before}表示句子中在文法符号前的字符,用综合属性\texttt{out}表示文法符号推出的字符总数,则翻译方案如下:
\begin{align*}
    S' \rightarrow\ & && \texttt{\{ }S\texttt{.before = 0; \}}\\
                    &S&& \\
    S \rightarrow\ & &&\texttt{\{ }L\texttt{.before = }S\texttt{.before + 1; \}}\\
                    &(L)&& \texttt{\{ }S\texttt{.out = }L\texttt{.out + 2; \}} \\
    S \rightarrow\ &a&& \texttt{\{ }S\texttt{.out = 1; print(}S\texttt{.before + 1); \}}\\
    L \rightarrow\ & && \texttt{\{ }L_1\texttt{.before = }L\texttt{.before; \}}\\
                    &L_1,&& \texttt{\{ }S\texttt{.before = }L\texttt{.before + }L_1\texttt{.out + 1; \}}\\
                    &S&& \texttt{\{ }L\texttt{.out = }L_1\texttt{.out + }S\texttt{.out + 1; \}} \\
    L \rightarrow\ & &&\texttt{\{ }S\texttt{.before = }L\texttt{.before; \}}\\
                    &S&& \texttt{\{ }L\texttt{.out = }S\texttt{.out; \}}
\end{align*}
\\

\section{Non-textbook Exercise}
\subsection*{(1)}
给出习题4.12(a)和(b)的翻译方案所对应的属性栈代码。

\paragraph{解}
对应的属性栈代码如下:
\newgeometry{left = 0.7cm, right = 0.7cm}
\columnseprule = 1pt
\begin{multicols}{2}
    \begin{align*}
        S' \rightarrow\ & &&\texttt{\{ }S\texttt{.depth = 0; \}}\\
                         &S&& \\
        S \rightarrow\ & &&\texttt{\{ }L\texttt{.depth = }S\texttt{.depth + 1; \}}\\
                        &(L)&& \\
        S \rightarrow\  &a&& \texttt{\{ print(}S\texttt{.depth); \}}\\
        L \rightarrow\ & &&\texttt{\{ }L_1\texttt{.depth = }L\texttt{.depth; \}}\\
                        &L_1,&& \texttt{\{ }S\texttt{.depth = }L\texttt{.depth; \}}\\
                        &S&& \\
        L \rightarrow\ & &&\texttt{\{ }S\texttt{.depth = }L\texttt{.depth; \}}\\
                        &S&&
    \end{align*}
    \\
    \begin{lstlisting}[language = C,
        keywordstyle=\bfseries\color{blue!70},
        commentstyle=\color{white!80!black},
        frame=shadowbox,
        framerule = 3pt,
        rulesepcolor = \color{white},
        xleftmargin = 2mm,
        rulecolor = \color{white},
        basicstyle=\ttfamily\small,
        lineskip = 11.5pt,
        escapeinside=``]
stack[top].depth = 0;

stack[top - 1].depth = stack[top - 2].depth + 1;

print(stack[top].depth);
// stack[top - 2].depth = stack[top - 2].depth;
stack[top].depth = stack[top - 2].depth;

// stack[top].depth = stack[top].depth;

    \end{lstlisting}
\end{multicols}
\begin{multicols}{2}
    \begin{align*}
        S' \rightarrow\ & && \texttt{\{ }S\texttt{.before = 0; \}}\\
                        &S&& \\
        S \rightarrow\ & &&\texttt{\{ }L\texttt{.before = }S\texttt{.before + 1; \}}\\
                        &(L)&& \texttt{\{ }S\texttt{.out = }L\texttt{.out + 2; \}} \\
        S \rightarrow\ &a&& \texttt{\{ }S\texttt{.out = 1; print(}S\texttt{.before + 1); \}}\\
        L \rightarrow\ & && \texttt{\{ }L_1\texttt{.before = }L\texttt{.before; \}}\\
                        &L_1,&& \texttt{\{ }S\texttt{.before = }L\texttt{.before + }L_1\texttt{.out + 1; \}}\\
                        &S&& \texttt{\{ }L\texttt{.out = }L_1\texttt{.out + }S\texttt{.out + 1; \}} \\
        L \rightarrow\ & &&\texttt{\{ }S\texttt{.before = }L\texttt{.before; \}}\\
                        &S&& \texttt{\{ }L\texttt{.out = }S\texttt{.out; \}}
    \end{align*}
    \\
    \begin{lstlisting}[language = C,
        keywordstyle=\bfseries\color{blue!70},
        commentstyle=\color{white!80!black},
        frame=shadowbox,
        framerule = 1pt,
        xleftmargin = 2mm,
        rulesepcolor = \color{white},
        rulecolor = \color{white},
        basicstyle=\ttfamily\scriptsize,
        lineskip = 16.5pt,
        escapeinside=``]
stack[top].depth = 0;

stack[top - 1].before = stack[top - 2].before + 1;
stack[top - 2].out = stack[top - 1].out + 2;
stack[top].out = 1; print(stack[top].before + 1);
// stack[top - 2].before = stack[top - 2].before;
stack[top].before = stack[top - 2].before + stack[top - 2].out + 1;
stack[top - 2].out = stack[top - 2].out + stack[top].out + 1;
// stack[top].before = stack[top].before;
// stack[top].out = stack[top].out;
    \end{lstlisting}
\end{multicols}
\restoregeometry

\subsection*{(2)}
给出习题4.12(a)和(b)的翻译方案所对应的YACC语义代码。

\paragraph{解}
\subparagraph{(a)}
翻译方案对应的语义代码如下:
\begin{lstlisting}[language = C,
    numbers=left,
    numberstyle=\small,
    keywordstyle=\bfseries\color{blue!70},
    commentstyle=\color{red!40!green!60!blue},
    frame=shadowbox,
    rulesepcolor=\color{red!20!green!30!blue!20},
    basicstyle=\ttfamily,
    breaklines,
    columns = fixed,
    escapeinside=``]
Start   :           { $1.depth = 0; }
         S
        ;
S       :           { $2.depth = $$.depth + 1; }
         '('L')'
        ;
S       :a          { printf("%d\n", $$.depth); }
        ;
L       :           { $1.depth = $$.depth; }
         L_1','     { $3.depth = $$.depth; }
         S
        ;
L_1     :           { $1.depth = $$.depth; }
         S
        ;
\end{lstlisting}
\subparagraph{(b)}
翻译方案对应的语义代码如下:
\begin{lstlisting}[language = C,
    numbers=left,
    numberstyle=\small,
    keywordstyle=\bfseries\color{blue!70},
    commentstyle=\color{red!40!green!60!blue},
    frame=shadowbox,
    rulesepcolor=\color{red!20!green!30!blue!20},
    basicstyle=\ttfamily,
    breaklines,
    columns = fixed,
    escapeinside=``]
Start   :           { $1.before = 0; }
         S
        ;
S       :           { $2.before = $$.before + 1; }
         '('L')'    { $$.out = $2.out + 2; }
        ;
S       :a          {
                        $$.out = 1;
                        printf("%d\n", $$.before + 1);
                    }
        ;
L       :           { $1.before = $$.before; }
         L_1','     { $3.before = $$.before + $1.out + 1; }
         S          { $$.out = $1.out + $3.out + 1; }
        ;
L_1     :           { $1.before = $$.before; }
         S          { $$.out = $1.out; }
        ;
\end{lstlisting}
\subsection*{(3)}
针对以下文法
\begin{align*}
    E \rightarrow &\ E\ \texttt{'>'}\ E \\
    |&\  E\ \texttt{'<'}\ E \\
    |&\  \textbf{number}
\end{align*}
设计语法制导定义,使之能计算诸如\texttt{1 < 2 < 3}的表达式值为True;而计算表达式\texttt{1 < 5 > 3}的值也为True。

\paragraph{解}
用综合属性$left$和$right$表示$E$推出的字符序列中最左和最右的数字,综合属性$bool$表示$E$推出的表达式的布尔值,语法制导的定义如下:
\begin{table}[H]
    \begin{tabular}{|l|l|}
    \hline
    \multicolumn{1}{|c|}{产生式} & 语义规则 \\ \hline
    $E'\rightarrow \ E$ & $print(E.bool);$ \\ \hline
    \multirow{2}{*}{$E \rightarrow \ E_1\ \texttt{'>'}\ E_2$} & \multirow{2}{*}{\begin{tabular}[c]{@{}l@{}}$E.left = E_1.left;\quad E.right = E_2.right;$\\ $E_1.bool = (E_1.right > E_2.left)\ ?\ \texttt{true}\ :\ \texttt{false}$\end{tabular}} \\
     &  \\ \hline
    \multirow{2}{*}{$E \rightarrow \ E_1\ \texttt{'<'}\ E_2$} & \multirow{2}{*}{\begin{tabular}[c]{@{}l@{}}$E.left = E_1.left;\quad E.right = E_2.right;$ \\ $E_1.bool = (E_1.right < E_2.left)\ ?\ \texttt{true}\ :\ \texttt{false}$\end{tabular}} \\
     &  \\ \hline
    $E \rightarrow \ \textbf{number}$ & $E.left = \textbf{number}.val;\quad E.right = \textbf{number}.val;\quad E.bool = \texttt{true};$ \\ \hline
    \end{tabular}
\end{table}
\end{document}