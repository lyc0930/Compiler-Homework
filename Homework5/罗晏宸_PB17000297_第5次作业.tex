\documentclass{article}
\usepackage[UTF8]{ctex}
\usepackage[T1]{fontenc}
\usepackage[utf8]{inputenc}
\usepackage{float}
\usepackage{placeins}
\usepackage{latexsym}
\usepackage{amsmath}
\usepackage{amsthm}
\usepackage{amssymb}
\usepackage{listings}
\usepackage{xcolor}
\usepackage{tikz}
\usepackage{tabularx}
\usepackage{multirow}
\usepackage{hyperref}
\usepackage{multicol}
\usepackage{geometry}
\title{Homework 5}
\author{PB17000297 罗晏宸}
\date{October 15 2019}

\hypersetup{
    colorlinks=true,
    linkcolor = black,
    urlcolor=blue!30!green,
}

\begin{document}
\maketitle

\section{Exercise 4.12}
文法如下:
\begin{align*}
    S &\rightarrow (L)\ |\ a \\
    L &\rightarrow L,\,S\ |\ S
\end{align*}
\subparagraph{(a)}
写一个翻译方案,它输出每个$a$的嵌套深度。例如,对于句子$(a,\,(a,\,a))$,输出的结果是$1\quad2\quad2$。
\subparagraph{(b)}
写一个翻译方案,它打印出每个$a$在句子中是第几个字符。例如,当句子是$(a,\,(a,\,(a,\,a),(a)))$时,打印的结果是$2\quad5\quad8\quad10\quad14$。

\paragraph{解}
\subparagraph{(a)}
用继承属性\texttt{depth}表示嵌套深度,则翻译方案如下:
\begin{align*}
    S' \rightarrow\ & &&\texttt{\{ }S\texttt{.depth = 0; \}}\\
                     &S&& \\
    S \rightarrow\ & &&\texttt{\{ }L\texttt{.depth = }S\texttt{.depth + 1; \}}\\
                    &(L)&& \\
    S \rightarrow\  &a&& \texttt{\{ print(}S\texttt{.depth); \}}\\
    L \rightarrow\ & &&\texttt{\{ }L_1\texttt{.depth = }L\texttt{.depth; \}}\\
                    &L_1,&& \texttt{\{ }S\texttt{.depth = }L\texttt{.depth; \}}\\
                    &S&& \\
    L \rightarrow\ & &&\texttt{\{ }S\texttt{.depth = }L\texttt{.depth; \}}\\
                    &S&&
\end{align*}
\subparagraph{(b)}
用继承属性\texttt{before}表示句子中在文法符号前的字符,用综合属性\texttt{out}表示文法符号推出的字符总数,则翻译方案如下:
\begin{align*}
    S' \rightarrow\ & && \texttt{\{ }S\texttt{.before = 0; \}}\\
                    &S&& \\
    S \rightarrow\ & &&\texttt{\{ }L\texttt{.before = }S\texttt{.before + 1; \}}\\
                    &(L)&& \texttt{\{ }S\texttt{.out = }L\texttt{.out + 2; \}} \\
    S \rightarrow\ &a&& \texttt{\{ }S\texttt{.out = 1; print(}S\texttt{.before + 1); \}}\\
    L \rightarrow\ & && \texttt{\{ }L_1\texttt{.before = }L\texttt{.before; \}}\\
                    &L_1,&& \texttt{\{ }S\texttt{.before = }L\texttt{.before + }L_1\texttt{.out + 1; \}}\\
                    &S&& \texttt{\{ }L\texttt{.out = }L_1\texttt{.out + }S\texttt{.out + 1; \}} \\
    L \rightarrow\ & &&\texttt{\{ }S\texttt{.before = }L\texttt{.before; \}}\\
                    &S&& \texttt{\{ }L\texttt{.out = }S\texttt{.out; \}}
\end{align*}
\\

\section{Non-textbook Exercise}
\subsection*{(1)}
给出习题4.12(a)和(b)的翻译方案所对应的属性栈代码。

\paragraph{解}
为了自底向上计算,必须确定继承属性在属性栈里的位置,为此引入标记非终结符$P$、$Q$和$R$及其继承属性$i$与综合属性$s$。更改后的翻译方案对应的属性栈代码如下:

\begin{figure}[H]
    \centering
    \columnseprule = 1pt
    \begin{multicols}{2}
    \begin{align*}
        S' \rightarrow\ &P&& \texttt{\{ }S\texttt{.depth = }P\texttt{.s; \}} \\
                        &S&& \\
        P \rightarrow\  &\varepsilon&& \texttt{\{ }P\texttt{.s = 0; \}} \\
        S \rightarrow\  & && \texttt{\{ }Q\texttt{.i = }S\texttt{.depth; \}} \\
                        &(&& \\
                        &Q&& \texttt{\{ }L\texttt{.depth = }Q\texttt{.s; \}} \\
                        &L)&& \\
        Q \rightarrow\  &\varepsilon&& \texttt{\{ }Q\texttt{.s = }Q\texttt{.i + 1; \}} \\
        S \rightarrow\  &a&& \texttt{\{ print(}S\texttt{.depth); \}}\\
        L \rightarrow\ & &&\texttt{\{ }L_1\texttt{.depth = }L\texttt{.depth; \}} \\
                        &L_1,&& \texttt{\{ }R\texttt{.i = }L\texttt{.depth; \}} \\
                        &R&& \texttt{\{ }S\texttt{.depth = }R\texttt{.s;\} } \\
                        &S&& \\
        R \rightarrow\  &\varepsilon&& \texttt{\{ }R\texttt{.s = }R\texttt{.i; \}} \\
        L \rightarrow\  & &&\texttt{\{ }S\texttt{.depth = }L\texttt{.depth; \}}\\
                        &S&&
    \end{align*}
    \begin{lstlisting}[language = C,
        keywordstyle=\bfseries\color{blue!70},
        commentstyle=\color{white!80!black},
        frame=shadowbox,
        framerule = 3pt,
        rulesepcolor = \color{white},
        xleftmargin = 2mm,
        rulecolor = \color{white},
        basicstyle=\ttfamily\small,
        lineskip = 11.5pt,
        escapeinside=``]
`\\[5pt]`

stack[ntop].val = 0;




stack[ntop].val = stack[top - 1].val + 1;
print(stack[top].val);




stack[ntop].val = stack[top - 2].val;


    \end{lstlisting}
    \end{multicols}
\end{figure}

\begin{figure}[H]
    \centering
    \columnseprule = 1pt
    \begin{multicols}{2}
    \begin{align*}
        S' \rightarrow\ &P&& \texttt{\{ }S\texttt{.before = }P\texttt{.s; \}}\\
                        &S&& \\
        P \rightarrow\  &\varepsilon&& \texttt{\{ }P\texttt{.s = 0; \}} \\
        S \rightarrow\  & && \texttt{\{ }Q\texttt{.i = }S\texttt{.before; \}} \\
                        &(&& \\
                        &Q&&\texttt{\{ }L\texttt{.before = }Q\texttt{.s; \}} \\
                        &L)&& \texttt{\{ }S\texttt{.out = }L\texttt{.out + 2; \}} \\
        Q \rightarrow\  &\varepsilon&& \texttt{\{ }Q\texttt{.s = }Q\texttt{.i + 1; \}} \\
        S \rightarrow\  &a&& \texttt{\{} \\
                        & && \qquad S\texttt{.out = 1;} \\
                        & && \qquad \texttt{print(}S\texttt{.before + 1);} \\
                        & && \texttt{\}}\\
        L \rightarrow\  & && \texttt{\{ }L_1\texttt{.before = }L\texttt{.before; \}}\\
                        &L_1,&& \texttt{\{ }R\texttt{.i = }L\texttt{.before + }L_1\texttt{.out; \}}\\
                        &R&& \texttt{\{ }S\texttt{.before = }R\texttt{.s; \}}\\
                        &S&& \texttt{\{ }L\texttt{.out} \\
                        & && \qquad \texttt{ = }L_1\texttt{.out + }S\texttt{.out + 1; \}} \\
        R \rightarrow\  &\varepsilon&& \texttt{\{ }R\texttt{.s = }R\texttt{.i + 1; \}} \\
        L \rightarrow\  & && \texttt{\{ }S\texttt{.before = }L\texttt{.before; \}} \\
                        &S&& \texttt{\{ }L\texttt{.out = }S\texttt{.out; \}}
    \end{align*}
    \\[7pt]
    \begin{lstlisting}[language = C,
        keywordstyle=\bfseries\color{blue!70},
        commentstyle=\color{white!80!black},
        frame=shadowbox,
        framerule = 1pt,
        xleftmargin = 2mm,
        rulesepcolor = \color{white},
        rulecolor = \color{white},
        basicstyle=\ttfamily\scriptsize,
        lineskip = 15pt,
        escapeinside=``]
`\\[10pt]`

stack[ntop].val = 0;



stack[ntop].val = stack[top - 1].val + 2;
stack[ntop].val = stack[top - 1].val + 1;

stack[ntop].val = 1;
print(stack[top].val + 1);





stack[ntop].val = stack[top - 3].val + stack[top].val + 1;
stack[ntop].val = stack[top - 2].val + stack[top - 1].val + 1;

stack[ntop].val = stack[top].val;
    \end{lstlisting}
    \end{multicols}
\end{figure}


\subsection*{(2)}
给出习题4.12(a)和(b)的翻译方案所对应的YACC语义代码。

\paragraph{解}
\subparagraph{(a)}
翻译方案对应的语义代码如下:
\begin{lstlisting}[language = C,
    numbers=left,
    numberstyle=\small,
    keywordstyle=\bfseries\color{blue!70},
    commentstyle=\color{red!40!green!60!blue},
    frame=shadowbox,
    rulesepcolor=\color{red!20!green!30!blue!20},
    basicstyle=\ttfamily,
    breaklines,
    columns = fixed,
    escapeinside=``]
Start   :P S
        ;
P       :           { $$ = 0; }
        ;
S       :'('Q L')'  { $$ = $0; }
        ;
Q       :           { $$ = $-1 + 1; }
        ;
S       :a          { $$ = $0; printf("%d\n", $$); }
        ;
L       :L','R S    { $$ = $0; }
        |S          { $$ = $0; }
        ;
R       :           { $$ = $-2; }
        ;
\end{lstlisting}
\subparagraph{(b)}
翻译方案对应的语义代码如下:
\begin{lstlisting}[language = C,
    numbers=left,
    numberstyle=\small,
    keywordstyle=\bfseries\color{blue!70},
    commentstyle=\color{red!40!green!60!blue},
    frame=shadowbox,
    rulesepcolor=\color{red!20!green!30!blue!20},
    basicstyle=\ttfamily,
    breaklines,
    columns = fixed,
    escapeinside=``]
Start   :P S
        ;
P       :           { $$ = 0; }
        ;
S       :'('Q L')'  { $$ = $3 + 2; }
        ;
Q       :           { $$ = $-1 + 1; }
        ;
S       :a          { $$ = 1; printf("%d\n", $0 + 1); }
        ;
L       :L','R S    { $$ = $1 + $4 + 1; }
        |S          { $$ = $1; }
        ;
R       :           { $$ = $-2 + $-1 + 1; }
        ;
\end{lstlisting}
\subsection*{(3)}
针对以下文法
\begin{align*}
    E \rightarrow &\ E\ \texttt{'>'}\ E \\
    |&\  E\ \texttt{'<'}\ E \\
    |&\  \textbf{number}
\end{align*}
设计语法制导定义,使之能计算诸如\texttt{1 < 2 < 3}的表达式值为True;而计算表达式\texttt{1 < 5 > 3}的值也为True。

\paragraph{解}
用综合属性$left$和$right$表示$E$推出的字符序列中最左和最右的数字,综合属性$bool$表示$E$推出的表达式的布尔值,语法制导的定义如下:
\begin{table}[H]
    \begin{tabular}{|l|l|}
    \hline
    \multicolumn{1}{|c|}{产生式} & \multicolumn{1}{c|}{语义规则} \\ \hline
    $E'\rightarrow \ E$ & $print(E.bool);$ \\ \hline
    $E \rightarrow \ E_1\ \texttt{'>'}\ E_2$ & \begin{tabular}[c]{@{}l@{}}$E.left = E_1.left;\ E.right = E_2.right;$\\ $E_1.bool =$ \\ $(E_1.right > E_2.left\ \&\&\ E_1.bool\ \&\&\ E_2.bool)\ ?\ \texttt{true}\ :\ \texttt{false}$\end{tabular} \\ \hline
    $E \rightarrow \ E_1\ \texttt{'<'}\ E_2$ & \begin{tabular}[c]{@{}l@{}}$E.left = E_1.left;\ E.right = E_2.right;$ \\ $E_1.bool =$\\ $(E_1.right < E_2.left\ \&\&\ E_1.bool\ \&\&\ E_2.bool)\ ?\ \texttt{true}\ :\ \texttt{false}$\end{tabular} \\ \hline
    $E \rightarrow \ \textbf{number}$ & $E.left = \textbf{number}.val;\ E.right = \textbf{number}.val;\ E.bool = \texttt{true};$ \\ \hline
    \end{tabular}
    \end{table}
\end{document}