\documentclass{article}
\usepackage[UTF8]{ctex}
\usepackage[T1]{fontenc}
\usepackage[utf8]{inputenc}
\usepackage{float}
\usepackage{placeins}
\usepackage{latexsym}
\usepackage{amsmath}
\usepackage{amsthm}
\usepackage{listings}
\usepackage{xcolor}
\usepackage{ulem}
\usepackage{multicol}
\usepackage{geometry}

\lstset
{
    basicstyle = \ttfamily,
    keywordstyle = \bfseries\color{blue!70},
    commentstyle = \songti \upshape,
    escapeinside=``,
    breaklines = true,
    breakatwhitespace = true,
    breakautoindent = true,
    texcl = true,
    showstringspaces = false,
    basewidth = 0.5em,
    flexiblecolumns,
    columns = fixed,
    frame = {},
}
\lstdefinestyle{C}
{
    language = C,
}
\lstdefinestyle{Assembler}
{
    language = [X86masm]Assembler,
    alsolanguage = C,
}

\title{Homework 8}
\author{PB17000297 罗晏宸}
\date{November 6 2019}


\begin{document}

\maketitle

\section*{Exercise 1}
针对以下C程序及其汇编代码:\uline{补全下划线部分的C语句或汇编代码}

\begin{multicols}{2}

    \begin{lstlisting}[style = C]
int test(int i, int j)
{
    int score;
    if(`\uline{~~~~~~~~~~~~~~~}`)
        score = 100;
    else
        score = 60;
    return score;
}// C程序
    \end{lstlisting}

    \begin{lstlisting}[style = Assembler, lineskip = 0.15em]
.text
.globl  test
    .type   test, @function
test:
    pushl   %ebp
    movl    %esp, %ebp
    `\uline{~~~~~~~~~~~~~~~~~~}`
    movl    8(%ebp), %eax
    cmpl    12(%ebp), %eax
    jle     .L4
    cmpl    $0, 12(%ebp)
    je      .L4
    cmpl    $10, 12(%ebp)
    jle     .L4
    cmpl    $0, 8(%ebp)
    je      .L4
    cmpl    $20, 8(%ebp)
    jg      .L3
.L4:
    cmpl    $100, 8(%ebp)
    jg      .L3
    cmpl    $99, 12(%ebp)
    jle     .L2
    cmpl    $40, 8(%ebp)
    jg      .L2
    cmpl    $20, 12(%ebp)
    jle     .L3
    cmpl    $-10, 8(%ebp)
    jge     .L3
    jmp     .L2
.L3:
    movl    $100, -4(%ebp)
    jmp     .L5
.L2:
    movl    $60, -4(%ebp)
.L5:
    `\uline{~~~~~~~~~~~~~}`
    leave
    ret
    \end{lstlisting}
\end{multicols}

% \newgeometry{left = 10em, right = 10em}

\begin{multicols}{2}
    \paragraph{解}
    补全后的代码如下
    \begin{lstlisting}[style = C]
int test(int i, int j)
{
    int score;
    if(`\color{red!20!green!60!blue}{\uline{i > j \&\& j != 0 \&\& j > 10 \&\& \\ i != 0 \&\& i > 20 || i > 100 || \\ (j >= 100 \&\& i <= 40) \&\& \\ (j <= 20 || i >= -10)}}`)
        score = 100;
    else
        score = 60;
    return score;
}// C程序
    \end{lstlisting}


    \begin{figure}[H]
        \centering
        \begin{lstlisting}[style = Assembler, lineskip = 0.2em]
.text
.globl  test
    .type   test, @function
test:
    pushl   %ebp
    movl    %esp, %ebp
    `\underline{\color{red!20!green!60!blue}{subl~~~~\$16, \%esp}}`
    movl    8(%ebp), %eax
    cmpl    12(%ebp), %eax
    jle     .L4
    cmpl    $0, 12(%ebp)
    je      .L4
    cmpl    $10, 12(%ebp)
    jle     .L4
    cmpl    $0, 8(%ebp)
    je      .L4
    cmpl    $20, 8(%ebp)
    jg      .L3
.L4:
    cmpl    $100, 8(%ebp)
    jg      .L3
    cmpl    $99, 12(%ebp)
    jle     .L2
    cmpl    $40, 8(%ebp)
    jg      .L2
    cmpl    $20, 12(%ebp)
    jle     .L3
    cmpl    $-10, 8(%ebp)
    jge     .L3
    jmp     .L2
.L3:
    movl    $100, -4(%ebp)
    jmp     .L5
.L2:
    movl    $60, -4(%ebp)
.L5:
    `\underline{\color{red!20!green!60!blue}{movl~~~~-4(\%ebp), \%eax}}`
    leave
    ret
        \end{lstlisting}
    \end{figure}
\end{multicols}

% \restoregeometry

\section*{Exercise 2}
给出生成如下C风格\lstinline{for}语句中间代码的翻译方案;假定代码生成的顺序不变。

\begin{align*}
    S \rightarrow \texttt{for(}E_1\texttt{; }E_2\texttt{; }E_3\texttt{)}\ S_1
\end{align*}

\begin{multicols}{2}
\paragraph{解}
首先给出中间代码的结构
\begin{lstlisting}[language = Pascal, alsolanguage = C]
    `$E_1$`
L1: if `$E_2$` goto L3
L2: `$E_3$`
    goto L1
L3: `$S_1$`
    goto L2
\end{lstlisting}

据此给出一个翻译方案如下
\begin{align*}
    S \rightarrow\  &\texttt{for(}E_1\texttt{; }M_1\ E_2\texttt{; }M_2\ E_3\texttt{)}\ N\ S_1 \\
                    &\texttt{\{} \\
                    &\qquad \texttt{emit('goto', }M_2\texttt{.next);} \\
                    &\qquad \texttt{backpatch(}E_2\texttt{.truelist, }N\texttt{.next);} \\
                    &\qquad \texttt{backpatch(}S_1\texttt{.nextlist, }M_2\texttt{.next);} \\
                    &\qquad \texttt{backpatch(}N\texttt{.nextlist, }M_1\texttt{.next);} \\
                    &\qquad S\texttt{.nextlist = }E_2\texttt{.falselist;} \\
                    &\texttt{\}}\\
    M \rightarrow\  &\varepsilon \\ &\texttt{\{ }M\texttt{.next = nextstmt; \}}\\
    N \rightarrow\  &\varepsilon \\
                    &\texttt{\{} \\
                    &\qquad N\texttt{.nextlist = makelist(nextstmt);} \\
                    &\qquad \texttt{emit('goto');} \\
                    &\qquad N\texttt{.next = nextstmt;} \\
                    &\texttt{\}}\\
\end{align*}
\end{multicols}
\end{document}

