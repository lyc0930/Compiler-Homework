\documentclass{article}
\usepackage[UTF8]{ctex}
\usepackage[T1]{fontenc}
\usepackage[utf8]{inputenc}
\usepackage{float}
\usepackage{placeins}
\usepackage{latexsym}
\usepackage{amsmath}
\usepackage{amsthm}
\usepackage{amssymb}
\usepackage{listings}
\usepackage{xcolor}
\usepackage{tikz}
\usepackage{tabularx}
\usepackage{multirow}
\usepackage{hyperref}

\title{Homework 4}
\author{PB17000297 罗晏宸}
\date{September 26 2019}

\hypersetup{
    colorlinks=true,
    linkcolor = black,
    urlcolor=blue!30!green,
}

\begin{document}
\maketitle

\section{Exercise 3.9}
为习题3.3的文法构造预测分析器(递归下降分析程序)。
\begin{equation*}
    S \rightarrow S\ \textbf{and}\ S\ |\ S\ \textbf{or}\ S\ |\ \textbf{not}\ S\ |\ \textbf{true}\ |\ \textbf{false}\ |\ \texttt{(}S\texttt{)}
\end{equation*}
\\

\paragraph{解}
该文法是二义的,等价的非二义文法为
\begin{align*}
    E &\rightarrow E\ \textbf{or}\ T\ |\ T \\
    T &\rightarrow T\ \textbf{and}\ F\ |\ F \\
    F &\rightarrow \textbf{not}\ F\ |\ \texttt{(}E\texttt{)}\ |\ \textbf{true}\ |\ \textbf{false}
\end{align*}
消除其中的左递归,得到
\begin{align*}
    E &\rightarrow TE' \\
    E' &\rightarrow \textbf{or}\ TE'\ |\ \varepsilon \\
    T &\rightarrow FT' \\
    T' &\rightarrow \textbf{and}\ FT'\ |\ \varepsilon \\
    F &\rightarrow \textbf{not}\ F\ |\ \texttt{(}E\texttt{)}\ |\ \textbf{true}\ |\ \textbf{false}
\end{align*}
列出文法中各非终结符的开始符号和后继符号集合
\begin{align*}
    &FIRST\texttt{(}E\texttt{)} = \big\{\ \textbf{not},\,\texttt{(},\,\textbf{true},\,\textbf{false}\ \big\} &&FOLLOW\texttt{(}E\texttt{)} = \big\{\ \texttt{)},\,\texttt{\$} \ \big\} \\
    &FIRST(E') = \big\{\ \textbf{or},\,\varepsilon \ \big\} &&FOLLOW(E') = \big\{\ \texttt{)},\,\texttt{\$}\ \big\} \\
    &FIRST(T) = \big\{\ \textbf{not},\,\texttt{(},\,\textbf{true},\,\textbf{false}\ \big\} &&FOLLOW(T) = \big\{\ \textbf{or},\,\texttt{)},\,\texttt{\$}\ \big\} \\
    &FIRST(T') = \big\{\ \textbf{and},\,\varepsilon \ \big\} &&FOLLOW(T') = \big\{\ \textbf{or},\,\texttt{)},\,\texttt{\$}\ \big\} \\
    &FIRST(F) = \big\{\ \textbf{not},\,\texttt{(},\,\textbf{true},\,\textbf{false} \ \big\} &&FOLLOW(F) = \big\{\ \textbf{and},\,\textbf{or},\,\texttt{)},\,\texttt{\$} \ \big\}
\end{align*}
对于产生式$E' \rightarrow \textbf{or}\ TE'\ |\ \varepsilon$
\begin{align*}
    &FIRST(\textbf{or}\ TE') \ \cap \  FIRST(\varepsilon) \\ =& \big\{\ \textbf{or}\ \big\} \ \cap \  \big\{\ \varepsilon\ \big\} \\
    =& \varnothing
    \\
    &FIRST(\textbf{or}\ TE') \ \cap \  FOLLOW(E') \\
    =& \big\{\ \textbf{or}\ \big\} \ \cap \  \big\{\ \texttt{)},\,\texttt{\$}\ \big\} \\
    =& \varnothing
\end{align*}
对于产生式$T' \rightarrow \textbf{and}\ FT'\ |\ \varepsilon$
\begin{align*}
    &FIRST(\textbf{and}\ FT') \ \cap \  FIRST(\varepsilon) \\ =& \big\{\ \textbf{and}\ \big\} \ \cap \  \big\{\ \varepsilon\ \big\} \\
    =& \varnothing
    \\
    &FIRST(\textbf{and}\ FT') \ \cap \  FOLLOW(T') \\ =& \big\{\ \textbf{and}\ \big\} \ \cap \  \big\{\ \textbf{or},\,\texttt{)},\,\texttt{\$}\ \big\} \\
    =& \varnothing
\end{align*}
对于产生式$F \rightarrow \textbf{not}\ F\ |\ \texttt{(}E\texttt{)}\ |\ \textbf{true}\ |\ \textbf{false}$
\begin{align*}
    &FIRST(\textbf{not}\ F) \ \cap \  FIRST(\texttt{(}E\texttt{)}) \ \cap \  FIRST(\textbf{true}) \ \cap \  FIRST(\textbf{false}) \\ = &\big\{\ \textbf{not}\ \big\} \ \cap \  \big\{\ \texttt{)}\ \big\} \ \cap \  \textbf{true} \ \cap \  \textbf{false}\\
    =& \varnothing
\end{align*}
因此该文法是LL(1)文法。为其构造递归下降预测分析器如下
\begin{lstlisting}[language = C,
    numbers=left,
    numberstyle=\small,
    keywordstyle=\bfseries\color{blue!70},
    commentstyle=\color{red!40!green!60!blue},
    frame=shadowbox,
    rulesepcolor=\color{red!20!green!30!blue!20},
    basicstyle=\ttfamily,
    breaklines,
    escapeinside=``]
    void match (terminal t)
    {
        if (lookahead == t)
            lookahead = nextToken();
        else
            error();
        return;
    }

    void `$E$`()
    {
        if ((lookahead == `$\textbf{not}$`) || (lookahead == '(') || (lookahead == `$\textbf{true}$`) || (lookahead == `$\textbf{false}$`))
        {
            `$T$`();
            `$E'$`();
        }
        else
            error();
        return;
    }

    void `$E'$`()
    {
        if (lookahead == `$\textbf{or}$`)
        {
            match(`\textbf{or}`);
            `$T$`();
            `$E'$`();
        }
        else if ((lookahead == ')') || (lookahead == '$'))
            return;
        else
            error();
        return;
    }

    void `$T$`()
    {
        if ((lookahead == `$\textbf{not}$`) || (lookahead == '(') || (lookahead == `$\textbf{true}$`) || (lookahead == `$\textbf{false}$`))
        {
            `$F$`();
            `$T'$`();
        }
        else
            error();
        return;
    }

    void `$T'$`()
    {
        if (lookahead == `$\textbf{and}$`)
        {
            match(`\textbf{and}`);
            `$F$`();
            `$T'$`();
        }
        else if (lookahead == `$\textbf{or}$`)
        {
            match(`\textbf{or}`);
            `$T$`();
            `$E'$`();
        }
        else if ((lookahead == ')') || (lookahead == '$'))
            return;
        else
            error();
        return;
    }

    void `$F$`()
    {
        if (lookahead == `\textbf{{not}}`)
        {
            match(`\textbf{{not}}`);
            `$F$`();
        }
        else if (lookahead == '(')
        {
            match('(');
            `$E$`();
            match(')');
        }
        else if (lookahead == `\textbf{true}`)
            match(`\textbf{true}`);
        else if (lookahead == `\textbf{false}`)
            match(`\textbf{false}`);
        else
            error();
        return;
    }
\end{lstlisting}

\section{Exercise 3.11}
构造下面文法的LL(1)分析表。
\begin{align*}
    S &\rightarrow aBS\ |\ bAS\ |\ \varepsilon \\
    A &\rightarrow bAA\ |\ a \\
    B &\rightarrow aBB\ |\ b
\end{align*}
\\

\paragraph{解}
列出文法中各非终结符的开始符号和后继符号集合
\begin{align*}
    &FIRST(S) = \big\{\  a,\,b,\,\varepsilon \ \big\} &&FOLLOW(S) = \big\{\  \texttt{\$} \ \big\} \\
    &FIRST(A) = \big\{\  a,\,b \ \big\} &&FOLLOW(A) = \big\{\  a,\,b,\,\texttt{\$} \ \big\} \\
    &FIRST(B) = \big\{\  a,\,b \ \big\} &&FOLLOW(B) = \big\{\  a,\,b,\,\texttt{\$} \ \big\}
\end{align*}
填写分析表如下
\begin{table}[H]
    \centering
    \caption{LL(1)分析表}
    \label{table:1}
    \begin{tabular}{|c|c|c|c|}
        \hline
         & $a$ & $b$ & $\texttt{\$}$ \\ \hline
        $S$ & $S \rightarrow aBS$ & $S \rightarrow bSS$ & $S \rightarrow \varepsilon$ \\ \hline
        $A$ & $A \rightarrow a$ & $A \rightarrow bAA$ &  \\ \hline
        $B$ & $B \rightarrow aBB$ & $B \rightarrow b$ &  \\ \hline
    \end{tabular}
\end{table}

\section{Exercise 3.16}
\subparagraph{(a)}
用习题3.1的文法构造$(a,\,(a,\,a)))$的最右推导,说出每个右句型的句柄。
\begin{align*}
    S &\rightarrow (L)\ |\ a \\
    L &\rightarrow L, \, S\ |\ S
\end{align*}
\\

\paragraph{解}
由此前构造分析树,可以得到句子$(a,\,(a,\,a))$的最右推导
\begin{align*}
    S &\Rightarrow_{\text{rm}} (L) \Rightarrow_{\text{rm}} (L,\,S) \Rightarrow_{\text{rm}} (L,\,(L)) \Rightarrow_{\text{rm}} (L,\,(L,\,S)) \Rightarrow_{\text{rm}} (L,\,(L,\,a)) \\
    &\Rightarrow_{\text{rm}} (L,\,(S,\,a))
    \Rightarrow_{\text{rm}} (L,\,(a,\,a)) \Rightarrow_{\text{rm}} (S,\,(a,\,a)) \\
    &\Rightarrow_{\text{rm}} (a,\,(a,\,a))
\end{align*}
给其中的$a$以下标,并给每个右句型的句柄添加下划线
\begin{align*}
    S &\Rightarrow_{\text{rm}} \underline{(L)} \\
    &\Rightarrow_{\text{rm}} (\underline{L,\,S}) \\
    &\Rightarrow_{\text{rm}} (L,\,\underline{(L)}) \\
    &\Rightarrow_{\text{rm}} (L,\,(\underline{L,\,S})) \\
    &\Rightarrow_{\text{rm}} (L,\,(L,\,\underline{a_3})) \\
    &\Rightarrow_{\text{rm}} (L,\,(\underline{S},\,a_3)) \\
    &\Rightarrow_{\text{rm}} (L,\,(\underline{a_2},\,a_3)) \\
    &\Rightarrow_{\text{rm}} (\underline{S},\,(a_2,\,a_3)) \\
    &\Rightarrow_{\text{rm}} (\underline{a_1},\,(a_2,\,a_3)) \\
\end{align*}

\section{Non-textbook Exercise}
\subparagraph{(1)}
删除以下文法G中的左递归,并由此得到文法G1。
\begin{table}[H]
    \centering
    \begin{tabular}{|c|c|}
        \hline
        & 文法G:$A$是开始符号 \\ \hline
        1 & $A \rightarrow Ba$ \\ \hline
        2 & $B \rightarrow dab$ \\ \hline
        3 & $B \rightarrow Cb$ \\ \hline
        4 & $C \rightarrow cB$ \\ \hline
        5 & $C \rightarrow Ac$ \\ \hline
    \end{tabular}
\end{table}
\subparagraph{(2)}
G1是否为LL(1)的文法?如不是,适当修改该文法G1,使之成为LL(1)的。

\paragraph{解}
\subparagraph{(1)}
用$A$的产生式$A \rightarrow Ba$代换$C \rightarrow Ac$中的$A$,再用产生式$B \rightarrow Cb$代换其中的$B$,用$B$的产生式$B \rightarrow Cb$代换$A \rightarrow Ba$中的$B$,再用产生式$C \rightarrow Ac$代换其中的$C$得到如下文法
\begin{table}[H]
    \centering
    \begin{tabular}{|c|c|}
        \hline
        & 文法G:$A$是开始符号 \\ \hline
        1 & $A \rightarrow Ba$ \\ \hline
        2 & $B \rightarrow dab$ \\ \hline
        3 & $B \rightarrow Cb$ \\ \hline
        4 & $C \rightarrow Cbac\ |\ dabac\ |\ cB$ \\ \hline
    \end{tabular}
\end{table}
删除其中的直接左递归,得到如下的文法
\begin{table}[H]
    \centering
    \caption{消除左递归的文法G1}
    \label{table:2}
    \begin{tabular}{|c|c|}
        \hline
        & 文法G:$A$是开始符号 \\ \hline
        1 & $A \rightarrow Ba$ \\ \hline
        2 & $B \rightarrow dab$ \\ \hline
        3 & $B \rightarrow Cb$ \\ \hline
        4 & $C \rightarrow dabacC'$ \\ \hline
        5 & $C \rightarrow cBC'$ \\ \hline
        6 & $C' \rightarrow bacC'$ \\ \hline
        7 & $C' \rightarrow \varepsilon$ \\ \hline
    \end{tabular}
\end{table}

\subparagraph{(2)}
列出文法中各非终结符的开始符号和后继符号集合
\begin{align*}
    &FIRST(A) = \big\{\  c,\,d \ \big\} &&FOLLOW(A) = \big\{\  \texttt{\$} \ \big\} \\
    &FIRST(B) = \big\{\  c,\,d \ \big\} &&FOLLOW(B) = \big\{\  a,\,b \ \big\} \\
    &FIRST(C) = \big\{\  c,\,d \ \big\} &&FOLLOW(C) = \big\{\  b \ \big\} \\
    &FIRST(C') = \big\{\  b,\,\varepsilon \ \big\} &&FOLLOW(C') = \big\{\  b,\,\texttt{\$} \ \big\}
\end{align*}
对于产生式$B \rightarrow dab\ |\ Cb$
\begin{align*}
    &FIRST(dab) \ \cap \  FIRST(Cb) \\
    =& \big\{\ d\ \big\} \ \cap \  FIRST(C) \\
    =& \big\{\ d\ \big\} \ \cap \  \big\{\ c,\,d\ \big\} \\
    =& \big\{\ d\ \big\}
\end{align*}
因此G1不是LL(1)文法。将$C$的产生式$C \rightarrow dabacC'\ |\ C \rightarrow cBC'$代入$B$的产生式$B \rightarrow dab\ |\ Cb$,将$B$的产生式$B \rightarrow dab\ |\ Cb$代入$A$的产生式$A \rightarrow Ba$,提左因子得到文法G2
\begin{table}[H]
    \centering
    \caption{消除左递归并提左因子的LL(1)文法G2}
    \label{table:3}
    \begin{tabular}{|c|c|}
        \hline
        & 文法G:$A$是开始符号 \\ \hline
        1 & $A \rightarrow dabaA'$ \\ \hline
        2 & $A \rightarrow cBbaA'$ \\ \hline
        3 & $A' \rightarrow cbaA'$ \\ \hline
        4 & $A' \rightarrow \varepsilon$ \\ \hline
        5 & $B \rightarrow cBC'b$ \\ \hline
        6 & $B \rightarrow dabB'$ \\ \hline
        7 & $B' \rightarrow acC'b$ \\ \hline
        8 & $B' \rightarrow \varepsilon$ \\ \hline
        9 & $C \rightarrow dabacC'$ \\ \hline
        10 & $C \rightarrow cBC'$ \\ \hline
        11 & $C' \rightarrow bacC'$ \\ \hline
        12 & $C' \rightarrow \varepsilon$ \\ \hline
    \end{tabular}
\end{table}
% 对于产生式$C \rightarrow dabaC'\ |\ cBC'$
% \begin{align*}
%     &FIRST(dabaC') \ \cap \  FIRST(cBC') \\
%     =& \big\{\ d\ \big\} \ \cap \  FIRST(c) \\
%     =& \big\{\ d\ \big\} \ \cap \  \big\{\ c\ \big\} \\
%     =& \varnothing
% \end{align*}
% 对于产生式$C' \rightarrow bacC'\ |\ \varepsilon$
% \begin{align*}
%     &FIRST(bacC') \ \cap \  FIRST(\varepsilon) \\
%     =& \big\{\ b\ \big\} \ \cap \  \big\{\ \varepsilon\ \big\} \\
%     =& \varnothing
%     \\
%     &FIRST(bacC') \ \cap \  FOLLOW(C') \\
%     =& \big\{\ b\ \big\} \ \cap \  \big\{\ \texttt{\$}\ \big\} \\
%     =& \varnothing
% \end{align*}
\end{document}