\documentclass{article}
\usepackage[UTF8]{ctex}
\usepackage[T1]{fontenc}
\usepackage[utf8]{inputenc}
\usepackage{float}
\usepackage{placeins}
\usepackage{latexsym}
\usepackage{amsmath}
\usepackage{amsthm}
\usepackage{listings}
\usepackage{xcolor}


\title{Homework 3}
\author{PB17000297 罗晏宸}
\date{September 16 2019}

\begin{document}
\maketitle

\section{Exercise 3.1}
考虑文法
\begin{align*}
    & S \rightarrow (L)\ |\ a \\
    & L \rightarrow L, \, S\ |\ S
\end{align*}
\subparagraph{(a)} 
建立句子$(a,\,(a,\,a))$和$(a,\,((a,\,a),\,(a,\,a)))$的分析树。
\subparagraph{(b)}
为(a)的两个句子构造最左推导。
\subparagraph{(c)}
为(a)的两个句子构造最右推导。
\subparagraph{(d)}
这个文法产生的语言是什么?
\\

\paragraph{解}
\subparagraph{(a)}
\subparagraph{(b)}
\subparagraph{(c)}
\subparagraph{(d)}
\\

\section{Exercise 3.2}
考虑文法
\begin{align*}
    S \rightarrow aSbS\ |\ bSaS\ |\ \varepsilon
\end{align*}
\subparagraph{(a)}
为句子$abab$构造两个不同的最左推导,以此说明该文法是二义的。
\subparagraph{(b)}
为$abab$构造对应的最右推导。
\subparagraph{(c)}
为$abab$构造对应的分析树。
\subparagraph{(d)}
这个文法产生的语言是什么?
\\

\paragraph{解}
\subparagraph{(a)}
\subparagraph{(b)}
\subparagraph{(c)}
\subparagraph{(d)}
\\

\section{Exercise 3}
阅读ANSI C 语法中从primary\_expression到 expression的产生式,了解C语言表达式的语法定义,并设计如下表格给出其中C算符的优先级和结合性。
\\
\paragraph{解}
\\

\section{Exercise 4}
阅读ANSI C 语法中declaration相关产生式,给出如下声明的分析树:
\begin{lstlisting}[language = C++, 
         keywordstyle=\bfseries\color{blue!70},
         commentstyle=\color{red!40!green!60!blue},
         frame=shadowbox,
         rulesepcolor=\color{red!20!green!30!blue!20},
         basicstyle=\ttfamily]
  void (*(*paa)[10])(int a);
\end{lstlisting}
\\
\paragraph{解}
\\
\end{document}