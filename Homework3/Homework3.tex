\documentclass{article}
\usepackage[UTF8]{ctex}
\usepackage[T1]{fontenc}
\usepackage[utf8]{inputenc}
\usepackage{float}
\usepackage{placeins}
\usepackage{latexsym}
\usepackage{amsmath}
\usepackage{amsthm}
\usepackage{listings}
\usepackage{xcolor}
\usepackage{tikz}
\usepackage{tabularx}
\usepackage{multirow}
\usepackage{longtable}

\title{Homework 3}
\author{PB17000297 罗晏宸}
\date{September 16 2019}

\begin{document}
\maketitle

\section{Exercise 3.1}
考虑文法
\begin{align*}
    & S \rightarrow (L)\ |\ a \\
    & L \rightarrow L, \, S\ |\ S
\end{align*}
\subparagraph{(a)} 
建立句子$(a,\,(a,\,a))$和$(a,\,((a,\,a),\,(a,\,a)))$的分析树。
\subparagraph{(b)}
为(a)的两个句子构造最左推导。
\subparagraph{(c)}
为(a)的两个句子构造最右推导。
\subparagraph{(d)}
这个文法产生的语言是什么?
\\

\paragraph{解}
\subparagraph{(a)}
两个分析树分别如图\ref{fig:1}和图\ref{fig:2}所示。
\begin{figure}
    \centering
    \begin{tikzpicture}[
        level/.style={sibling distance=20mm},
        scale = 1.0, 
        transform shape]
        \node (a) {$S$}
            child {node (b1) {$($}}
            child {node (b2) {$L$}
                child {node (c1) {$L$}
                    child {node (d1) {$S$}
                        child {node (e1) {$a$}}
                    }
                }
                child {node (c2) {$,$}}
                child {node (c3) {$S$}
                    child {node (d2) {$($}}
                    child {node (d3) {$L$}
                        child {node (e2) {$L$}
                            child {node (f1) {$S$}
                                child {node (g1) {$a$}}
                            }
                        }
                        child {node (e3) {$,$}}
                        child {node (e4) {$S$}
                            child {node (f2) {$a$}}
                        }
                    }
                    child {node (d4) {$)$}}
                }
            }
            child {node (b3) {$)$}}
    \end{tikzpicture}
    \caption{句子$(a,\,(a,\,a))$的分析树}
    \label{fig:1}
\end{figure}

\begin{figure}
    \centering
    \begin{tikzpicture}[
        level 1/.style={sibling distance=25mm},
        level 2/.style={sibling distance=25mm},
        level 3/.style={sibling distance=25mm},
        level 4/.style={sibling distance=25mm},
        level 5/.style={sibling distance=20mm},
        level 6/.style={sibling distance=20mm},
        level 7/.style={sibling distance=20mm},
        level 8/.style={sibling distance=20mm},
        level 9/.style={sibling distance=20mm},
        scale = 1.0, 
        transform shape]
        \node (a) {$S$}
            child {node (b1) {$($}}
            child {node (b2) {$L$}
                child {node (c1) {$L$}
                    child {node (d1) {$S$}
                        child {node (e1) {$a$}}
                    }
                }
                child {node (c2) {$,$}}
                child {node (c3) {$S$}
                    child {node (d2) {$($}}
                    child {node (d3) {$L$}
                        child {node (e2) {$L$}
                            child {node (f1) {$S$}
                                child {node (g1) {$($}}
                                child {node (g2) {$L$}
                                    child {node (h1) {$L$}
                                        child {node (i1) {$S$}
                                            child {node (j1) {$a$}}
                                        }
                                    }
                                    child {node (h2) {$,$}}
                                    child {node (h3) {$S$}
                                        child {node (i2) {$a$}}
                                    }
                                }
                                child {node (g3) {$)$}}
                            }
                        }
                        child {node (e3) {$,$}}
                        child {node (e4) {$S$}
                             child {node (f2) {$($}}
                                child {node (g4) {$L$}
                                    child {node (h4) {$L$}
                                        child {node (i3) {$S$}
                                            child {node (j2) {$a$}}
                                        }
                                    }
                                    child {node (h5) {$,$}}
                                    child {node (h6) {$S$}
                                        child {node (i4) {$a$}}
                                    }
                                }
                                child {node (g5) {$)$}}
                        }
                    }
                    child {node (d4) {$)$}}
                }
            }
            child {node (b3) {$)$}}
    \end{tikzpicture}
    \caption{句子$(a,\,((a,\,a),\,(a,\,a)))$的分析树}
    \label{fig:2}
\end{figure}

\subparagraph{(b)}
由分析树,句子$(a,\,(a,\,a))$的最左推导为

\begin{align*}
    S &\Rightarrow_{\text{lm}} (L) \Rightarrow_{\text{lm}} (L,\,S) \Rightarrow_{\text{lm}} (S,\,S) \Rightarrow_{\text{lm}} (a,\,S) \Rightarrow_{\text{lm}} (a,\,(L)) \\ 
    &\Rightarrow_{\text{lm}} (a,\,(L,\,S))
    \Rightarrow_{\text{lm}} (a,\,(S,\,S)) \Rightarrow_{\text{lm}} (a,\,(a,\,S)) \\
    &\Rightarrow_{\text{lm}} (a,\,(a,\,a))
\end{align*}

句子$(a,\,((a,\,a),\,(a,\,a)))$的最左推导为

\begin{align*}
    S &\Rightarrow_{\text{lm}} (L) \Rightarrow_{\text{lm}} (L,\,S) \Rightarrow_{\text{lm}} (S,\,S) \Rightarrow_{\text{lm}} (a,\,S) \Rightarrow_{\text{lm}} (a,\,(L)) \Rightarrow_{\text{lm}} (a,\,(L,\,S)) \\
    &\Rightarrow_{\text{lm}} (a,\,(S,\,S)) \Rightarrow_{\text{lm}} (a,\,((L),\,S)) \Rightarrow_{\text{lm}} (a,\,((L,\,S),\,S)) \Rightarrow_{\text{lm}} (a,\,((S,\,S),\,S)) \\
    &\Rightarrow_{\text{lm}} (a,\,((a,\,S),\,S))
    \Rightarrow_{\text{lm}} (a,\,((a,\,a),\,S)) \Rightarrow_{\text{lm}} (a,\,((a,\,a),\,(L))) \\
    &\Rightarrow_{\text{lm}} (a,\,((a,\,a),\,(L,\,S)))
    \Rightarrow_{\text{lm}} (a,\,((a,\,a),\,(S,\,S))) \Rightarrow_{\text{lm}} (a,\,((a.\,a),\,(a,\,S))) \\
    &\Rightarrow_{\text{lm}} (a,\,((a,\,a),\,(a,\,a)))
\end{align*}

\subparagraph{(c)}
由分析树,句子$(a,\,(a,\,a))$的最右推导为

\begin{align*}
    S &\Rightarrow_{\text{lm}} (L) \Rightarrow_{\text{lm}} (L,\,S) \Rightarrow_{\text{lm}} (L,\,(L)) \Rightarrow_{\text{lm}} (L,\,(L,\,S)) \Rightarrow_{\text{lm}} (L,\,(L,\,a)) \\ 
    &\Rightarrow_{\text{lm}} (L,\,(S,\,a))
    \Rightarrow_{\text{lm}} (L,\,(a,\,a)) \Rightarrow_{\text{lm}} (S,\,(a,\,a)) \\
    &\Rightarrow_{\text{lm}} (a,\,(a,\,a))
\end{align*}

句子$(a,\,((a,\,a),\,(a,\,a)))$的最右推导为

\begin{align*}
    S &\Rightarrow_{\text{lm}} (L) \Rightarrow_{\text{lm}} (L,\,S) \Rightarrow_{\text{lm}} (L,\,(L)) \Rightarrow_{\text{lm}} (L,\,(L,\,S)) \Rightarrow_{\text{lm}} (L,\,(L,\,(L))) \\ &\Rightarrow_{\text{lm}} (L,\,(L,\,(L,\,S)))
    \Rightarrow_{\text{lm}} (L,\,(L,\,(L,\,a))) \Rightarrow_{\text{lm}} (L,\,(L,\,(S,\,a)) \\ &\Rightarrow_{\text{lm}} (L,\,(L,\,(a,\,a)) \Rightarrow_{\text{lm}} (L,\,(S,\,(a,\,a))
     \Rightarrow_{\text{lm}} (L,\,((L),\,(a,\,a))) \\
    &\Rightarrow_{\text{lm}} (L,\,((L,\,S),\,(a,\,a)))
    \Rightarrow_{\text{lm}} (L,\,((L,\,a),\,(a,\,a))) \Rightarrow_{\text{lm}} (L,\,((S.\,a),\,(a,\,a))) \\
    &\Rightarrow_{\text{lm}} (L,\,((a,\,a),\,(a,\,a)))
    \Rightarrow_{\text{lm}} (S,\,((a,\,a),\,(a,\,a))) \\
    &\Rightarrow_{\text{lm}} (a,\,((a,\,a),\,(a,\,a)))
\end{align*}

\subparagraph{(d)}
这个文法描述的语言是“在配对括号内由逗号分割的二元组字符串,其中组的元素是$a$或一个由此语言描述的串”。
\\

\section{Exercise 3.2}
考虑文法

\begin{align*}
    S \rightarrow aSbS\ |\ bSaS\ |\ \varepsilon
\end{align*}

\subparagraph{(a)}
为句子$abab$构造两个不同的最左推导,以此说明该文法是二义的。
% \subparagraph{(b)}
% 为$abab$构造对应的最右推导。
% \subparagraph{(c)}
% 为$abab$构造对应的分析树。
% \subparagraph{(d)}
% 这个文法产生的语言是什么?
\\

\paragraph{解}
\subparagraph{(a)}
句子$abab$的两个不同的最左推导如下:

\begin{align*}
    S &\Rightarrow_{\text{lm}} aSbS \Rightarrow_{\text{lm}} abS \Rightarrow_{\text{lm}} abaSbS \Rightarrow_{\text{lm}} ababS \Rightarrow_{\text{lm}} abab \\
    S &\Rightarrow_{\text{lm}} aSbS \Rightarrow_{\text{lm}} abSaSbS \Rightarrow_{\text{lm}} abaSbS \Rightarrow_{\text{lm}} ababS \Rightarrow_{\text{lm}} abab
\end{align*}

因此这个文法是二义的。
% \subparagraph{(b)}
% 句子$abab$的一个最右推导为
% \begin{align*}
%     S \Rightarrow_{\text{lm}} aSbS \Rightarrow_{\text{lm}} aSbaSbS \Rightarrow_{\text{lm}} aSbaSb \Rightarrow_{\text{lm}} aSbab \Rightarrow_{\text{lm}} abab
% \end{align*}
% \subparagraph{(c)}
% 句子$abab$的一个分析树如图\ref{fig:3}所示。
% \begin{figure}
%     \centering
%     \begin{tikzpicture}[
%         level/.style={sibling distance=20mm},
%         scale = 1.0, 
%         transform shape]
%         \node (a) {$S$}
%             child {node (b1) {$a$}}
%             child {node (b2) {$S$}
%                 child {node (c1) {$\varepsilon$}}
%             }
%             child {node (b3) {$b$}}
%             child {node (b4) {$S$}
%                 child {node (c2) {$a$}}
%                 child {node (c3) {$S$}
%                     child {node (d1) {$\varepsilon$}}
%                 }
%                 child {node (c4) {$b$}}
%                 child {node (c5) {$S$}
%                     child {node (d2) {$\varepsilon$}}
%                 }
%             }
%     \end{tikzpicture}
%     \caption{句子$abab$的一个分析树}
%     \label{fig:3}
% \end{figure}
% \subparagraph{(d)}
% 这个文法产生的语言是“$a$和$b$个数相等的串的集合”
\\

\section{Exercise 3}
阅读ANSI C语法中从primary\_expression到expression的产生式,了解C语言表达式的语法定义,并设计如下表格给出其中C算符的优先级和结合性。
\FloatBarrier

\begin{figure}[H]
    \centering
    \begin{tabular}{cll}
        \hline
        \textbf{优先级} & \multicolumn{1}{c}{\textbf{算符(组)}} & \multicolumn{1}{c}{\textbf{结合性}} \\ \hline
1 &  & 
    \end{tabular}
\end{figure}

\FloatBarrier
\\
\paragraph{解}
如表\ref{table:1}所示。

\begin{longtable}{|c|c|c|}
        
    \caption{C语言算符的优先级与结合性(部分)}
    \label{table:1} \\
    
    \hline
    \textbf{优先级} & \textbf{算符(组)} & \textbf{结合性} \\ 
    \hline 
    \endfirsthead
    
    \multicolumn{3}{r}{{\kaishu 续表}} \\ 
    \hline
    \textbf{优先级} & \textbf{算符(组)} & \textbf{结合性} \\ 
    \hline 
    \endhead
    
    \hline
    \endfoot
    
    \hline
    \endlastfoot
    
    {\multirow{4}{*}{1}} 
     & \texttt{{[}{]}} & \multirow{4}{*}{左结合} \\ \cline{2-2}
     & \texttt{()} &  \\ \cline{2-2}
     & \texttt{.} &  \\ \cline{2-2}
     & \texttt{->} &  \\ \hline
    \multirow{9}{*}{2}
     & \texttt{-}(负号) & \multirow{9}{*}{右结合} \\ \cline{2-2}
     & (\texttt{type name}) &  \\ \cline{2-2}
     & \texttt{++} &  \\ \cline{2-2}
     & \texttt{--} &  \\ \cline{2-2}
     & \texttt{*}(取值) &  \\ \cline{2-2}
     & \texttt{\&}(取地址) &  \\ \cline{2-2}
     & \texttt{!} &  \\ \cline{2-2}
     & \texttt{$\sim$} &  \\ \cline{2-2}
     & \texttt{sizeof} &  \\ \hline
    \multirow{3}{*}{3} 
     & \texttt{/} & \multirow{18}{*}{左结合} \\ \cline{2-2}
     & \texttt{*} &  \\ \cline{2-2}
     & \texttt{\%} &  \\ \cline{1-2}
    \multirow{2}{*}{4}
     & \texttt{+} &  \\ \cline{2-2}
     & \texttt{-} &  \\ \cline{1-2}
    \multirow{2}{*}{5} 
     & \texttt{{<}{<}} &  \\ \cline{2-2}
     & \texttt{{>}{>}} &  \\ \cline{1-2}
    \multirow{4}{*}{6} 
     & \texttt{<} &  \\ \cline{2-2}
     & \texttt{<=} &  \\ \cline{2-2}
     & \texttt{>} &  \\ \cline{2-2}
     & \texttt{>=} &  \\ \cline{1-2}
    \multirow{2}{*}{7} 
     & \texttt{==} &  \\ \cline{2-2}
     & \texttt{!=} &  \\ \cline{1-2}
    8 & \texttt{\&} &  \\ \cline{1-2}
    9 & \texttt{!} &  \\ \cline{1-2}
    10 & \texttt{|} &  \\ \cline{1-2}
    11 & \texttt{\&\&} &  \\ \cline{1-2}
    12 & \texttt{||} &  \\ \hline
    13 & \texttt{?:} & \multirow{4}{*}{右结合} \\ \cline{1-2}
    \multirow{3}{*}{14} 
     & \texttt{=} &  \\ \cline{2-2}
     & \texttt{*=} &  \\ \cline{2-2}
     & \texttt{/=} &  \\ \cline{2-2}
    \multirow{9}{*}{14}
     & \texttt{\%=} & \multirow{9}{*}{右结合} \\ \cline{2-2}
     & \texttt{+=} &  \\ \cline{2-2}
     & \texttt{-=} &  \\ \cline{2-2}
     & \texttt{{<}<=} & \\ \cline{2-2}
     & \texttt{{>}>=} &\\ \cline{2-2}
     & \texttt{=} &  \\ \cline{2-2}
     & \texttt{\&=} &  \\ \cline{2-2}
     & \texttt{!=} &  \\ \cline{2-2}
     & \texttt{|=} &  \\ \hline
    15 & \texttt{,} & 左结合 \\ \hline
\end{longtable}

\\

\section{Exercise 4}
阅读ANSI C语法中declaration相关产生式,给出如下声明的分析树:
\begin{lstlisting}[language = C++, 
         keywordstyle=\bfseries\color{blue!70},
         commentstyle=\color{red!40!green!60!blue},
         frame=shadowbox,
         rulesepcolor=\color{red!20!green!30!blue!20},
         basicstyle=\ttfamily]
  void (*(*paa)[10])(int a);
\end{lstlisting}
\\

\paragraph{解}
声明的分析树如图\ref{fig:4}所示

\begin{figure}
    \centering
    \begin{tikzpicture}[
        level 1/.style ={sibling distance = 4cm},
        level 5/.style ={sibling distance = 5cm, scale = 0.9},
        level 6/.style ={sibling distance = 2cm},
        level 7/.style ={sibling distance = 5cm},
        level 8/.style ={sibling distance = 2.5cm, scale = 0.8},
        level 9/.style ={sibling distance = 1.5cm},
        level 10/.style ={sibling distance = 2.2cm},
        level distance = 1cm,
        scale = 0.9, 
        transform shape]
        \node {declaration}
            child {node {declaration\_specifiers}
                child{node {type\_specifier}
                    child {node {VOID}
                        child {node [scale = 1.78] {\texttt{void}}}
                    }        
                }
            }
            child{node {init\_declarator\_list}
                child{node {init\_declarator}
                    child{node {declarator}
                        child{node {direct\_declarator}
                            child{node {direct\_declarator}
                                child{node [scale = 2.5] {\texttt{(}}}
                                child{node {declarator}
                                    child{node {pointer}
                                        child{node [scale = 3.0] {\texttt{*}}}
                                    }
                                    child{node {direct\_declarator}
                                        child{node {direct\_declarator}
                                            child{node [scale = 2.5] {\texttt{(}}}
                                            child{node {declarator}
                                                child{node {pointer}
                                                    child{node  [scale = 3.0]  {\texttt{*}}}
                                                }
                                                child{node {direct\_declarator}
                                                    child{node {IDENTIFIER}
                                                        child{node [scale = 2.5] {\texttt{paa}}}
                                                    }
                                                }
                                            }
                                            child{node [scale = 2.5] {\texttt{)}}}
                                        }
                                        child{node [scale = 2.5] {\texttt{[}}}
                                        child{node {constant\_expression}
                                            child{node {conditional\_expression}
                                                child{node {logical\_or\_expression}
                                                    child{node {logical\_and\_expression}
                                                        child{node {inclusive\_or\_expression}
                                                            child{node {exclusive\_or\_expression}
                                                                child{node {and\_expression}
                                                                    child{node {equality\_expression}
                                                                        child{node {relational\_expression}
                                                                            child{node {shift\_expression}
                                                                                child{node {additive\_expression}
                                                                                    child{node {multiplicative\_expression}
                                                                                        child{node {cast\_expression}
                                                                                            child{node {unary\_expression}
                                                                                                child{node {postfix\_expression}
                                                                                                    child{node {primary\_expression}
                                                                                                        child{node {CONSTANT}
                                                                                                            child{node [scale = 2.5] {\texttt{10}}}
                                                                                                        }
                                                                                                    }
                                                                                                }
                                                                                            }
                                                                                        }
                                                                                    }
                                                                                }
                                                                            }
                                                                        }
                                                                    }
                                                                }
                                                            }
                                                        }
                                                    }
                                                }
                                            }
                                        }
                                        child{node [scale = 2.5] {\texttt{]}}}
                                    }
                                }
                                child{node [scale = 2.5] {\texttt{)}}}
                            }
                            child{node [scale = 2.5] {\texttt{(}}}
                            child{node {parameter\_type\_list}
                                child{node {parameter\_list}
                                    child {node {parameter\_declaration}
                                        child [sibling distance =3.5cm] {node {declaration\_specifiers}
                                            child{node {type\_specifier}
                                                child{node {INT}
                                                    child{node [scale = 2.5] {\texttt{int}}}
                                                }
                                            }
                                        }
                                        child [sibling distance =3.5cm] {node {declarator}
                                            child{node {direct\_declarator}
                                                child{node {IDENTIFIER}
                                                    child{node [scale = 2.5] {\texttt{a}}}
                                                }
                                            }
                                        }
                                    }
                                }
                            }
                            child{node [scale = 2.5] {\texttt{)}}}
                        }
                    }
                }
            }
            child{node  [scale = 2.5] {\texttt{;}}}
    \end{tikzpicture}
    \caption{声明\texttt{void (*(*paa)[10])(int a);}的分析树}
    \label{fig:4}
\end{figure}

\\
\end{document}