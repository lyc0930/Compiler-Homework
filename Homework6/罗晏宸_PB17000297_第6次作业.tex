\documentclass{article}
\usepackage[UTF8]{ctex}
\usepackage[T1]{fontenc}
\usepackage[utf8]{inputenc}
\usepackage{float}
\usepackage{placeins}
\usepackage{latexsym}
\usepackage{amsmath}
\usepackage{amsthm}
\usepackage{listings}
\usepackage{xcolor}
\usepackage{multicol}
\usepackage{geometry}
\usepackage{mdframed}
\usepackage{moreverb}
\usepackage{ulem}
\usepackage{tikz}
\usetikzlibrary{positioning}
\usetikzlibrary[arrows]
\lstset
{
    basicstyle=\ttfamily,
    keywordstyle=\bfseries\color{blue!70},
    escapeinside=``,
    breaklines = true,
    breakatwhitespace = true,
    breakautoindent = true,
    texcl = true,
    showstringspaces = false,
    flexiblecolumns,
    columns = fixed,
    basewidth = 0.5em,
    lineskip = -0.1em,
    frame = {},
    aboveskip = 0pt
}
\lstdefinestyle{C}
{
    language = C,
}
\lstdefinestyle{Assembler}
{
    language = [X86masm]Assembler,
    alsolanguage = C,
}
\mdfdefinestyle{Default}
{
    innertopmargin = 0pt,
    innerbottommargin = 0pt,
    innerleftmargin = 3pt,
    innerrightmargin = 3pt,
    linewidth = 0.5pt,
}

\title{Homework 6}
\author{PB17000297 罗晏宸}
\date{October 23 2019}

\newcommand{\CodeBlock}[2]{
    \begin{figure}[H]
        \begin{mdframed}[
            style = Default,
        ]
            \lstinputlisting[
                style = #1,
            ]{#2}
        \end{mdframed}
    \end{figure}
}

\begin{verbatimwrite}{Ex1.c}
#include<stdio.h>
union var{
    char c[5];
    int i;
};
int main(){
    union var data;
    char *c;
    data.c[0] = '2';
    data.c[1] = '0';
    data.c[2] = '1';
    data.c[3] = '6';
    data.c[4] = '\0';
    c = (char*)&data;
    printf("%x %s\n", data.i, c);
    return 0;
}
// \songti \upshape 第一题C程序
\end{verbatimwrite}

\begin{verbatimwrite}{Ex1_Unfilled.s}
    .section .rodata
.LC0:
    .string "%x %s\n"
    .text
.globl main
    .type   main, @function
main:
    `\underline{~~~~~~~~~~~~~~~~~~~~~~}`
    movl    %esp, %ebp
    subl    $40, %esp
    andl    $-16, %esp
    movl    $0, %eax
    subl    %eax, %esp
    movb    $50, -24(%ebp)
    `\underline{~~~~~~~~~~~~~~~~~~~~~~}`
    `\underline{~~~~~~~~~~~~~~~~~~~~~~}`
    `\underline{~~~~~~~~~~~~~~~~~~~~~~}`
    `\underline{~~~~~~~~~~~~~~~~~~~~~~}`
    `\underline{~~~~~~~~~~~~~~~~~~~~~~}`
    movl    %eax, -28(%ebp)
    `\underline{~~~~~~~~~~~~~~~~~~~~~~}`
    `\underline{~~~~~~~~~~~~~~~~~~~~~~}`
    `\underline{~~~~~~~~~~~~~~~~~~~~~~}`
    pushl   $.LC0
    call    printf
    addl    $16, %esp
    `\underline{~~~~~~~~~~~~~~~~~~~~~~}`
    leave
    ret
// \songti 第一题汇编程序
\end{verbatimwrite}

\begin{verbatimwrite}{Ex1_Filled.s}
.section .rodata
.LC0:
    .string "%x %s\n"
    .text
.globl main
    .type   main, @function
main:
    `\underline{\mdseries \ttfamily \color{red!20!green!60!blue}{pushl~~~\%ebp}}`
    movl    %esp, %ebp
    subl    $40, %esp
    andl    $-16, %esp
    movl    $0, %eax
    subl    %eax, %esp
    movb    $50, -24(%ebp)
    `\underline{\mdseries \ttfamily \color{red!20!green!60!blue}{movb~~~~\$48, -23(\%ebp)}}`
    `\underline{\mdseries \ttfamily \color{red!20!green!60!blue}{movb~~~~\$49, -22(\%ebp)}}`
    `\underline{\mdseries \ttfamily \color{red!20!green!60!blue}{movb~~~~\$54, -21(\%ebp)}}`
    `\underline{\mdseries \ttfamily \color{red!20!green!60!blue}{movb~~~~\$0, -20(\%ebp)}}`
    `\underline{\mdseries \ttfamily \color{red!20!green!60!blue}{leal~~~~-24(\%ebp), \%eax}}`
    movl    %eax, -28(%ebp)
    `\underline{\mdseries \ttfamily \color{red!20!green!60!blue}{movl~~~~-24(\%ebp), \%eax}}`
    `\underline{\mdseries \ttfamily \color{red!20!green!60!blue}{movl~~~~-28(\%ebp), \%edx}}`
    `\underline{\mdseries \ttfamily \color{red!20!green!60!blue}{movl~~~~\%edx, \%esp}}`
    pushl   $.LC0
    call    printf
    addl    $16, %esp
    `\underline{\mdseries \ttfamily \color{red!20!green!60!blue}{popl~~~~\%esp}}`
    leave
    ret
// \songti 补全后的第一题汇编程序
\end{verbatimwrite}

\begin{verbatimwrite}{Ex2.c}
#define N 2
// \upshape \ttfamily \#define N 11
typedef struct POINT {
    int x, y ;
    char z[ N ];
    struct POINT *next;
} DOT;
void f(DOT p)
{
    p.x = 100;
    p.y = sizeof(p);
    p.z[1] = 'A';
    f(*(p.next));
}
// \songti \upshape 第二题C程序
\end{verbatimwrite}

\begin{verbatimwrite}{Ex2_N=2.s}
    .file "test1.c"
    .text
.globl f
    .type   f,@function
f:
    pushl   %ebp
    movl    %esp, %ebp
    movl    $100, 8(%ebp)
    movl    $16, 12(%ebp)
    movb    $65, `\underline{~~~~~~~~}`
    movl    `\underline{~~~~~~~}`, %eax
    pushl   `\underline{~~~~~~~~~~~~~}`
    pushl   `\underline{~~~~~~~~~~~~~}`
    pushl   `\underline{~~~~~~~~~~~~~}`
    pushl   `\underline{~~~~~~~~~~~~~}`
    call    f
    addl    $16, %esp
    leave
    ret
// \songti \upshape 当N=2时,生成的汇编代码片段。
\end{verbatimwrite}

\begin{verbatimwrite}{Ex2_N=11.s}
    .file "test1.c"
    .text
.globl f
    .type   f,@function
f:
    pushl   %ebp
    movl    %esp, %ebp
    pushl   %edi
    pushl   %esi
    movl    $100, 8(%ebp)
    movl    `\underline{~~~}`, 12(%ebp)
    movb    $65, `\underline{~~~~~~~}`
    subl    $8, %esp
    movl    `\underline{~~~~~~}`, %eax
    subl    $24, %esp
    movl    %esp, %edi
    movl    %eax, %esi
    cld
    movl    `\underline{~~~~~~}`, %eax
    movl    %eax, %ecx
    rep
    movsl
    call    f
    addl    $32, %esp
    leal    -8(%ebp), %esp
    popl    %esi
    popl    %edi
    leave
    ret
// \songti \upshape rep movsl为数据传送指令,即,由源地址esi开始的ecx个字的数据传送到由edi指示的目的地址。
// \songti \upshape 当N=11时,生成的汇编代码片段
\end{verbatimwrite}

\begin{verbatimwrite}{Ex2_N=2_Filled.s}
    .file "test1.c"
    .text
.globl f
    .type   f,@function
f:
    pushl   %ebp
    movl    %esp, %ebp
    movl    $100, 8(%ebp)
    movl    $16, 12(%ebp)
    movb    $65, `\underline{\mdseries \ttfamily \color{red!20!green!60!blue}{17(\%ebp)}}`
    movl    `\underline{\mdseries \ttfamily \color{red!20!green!60!blue}{20(\%ebp)}}`, %eax
    pushl   `\underline{\mdseries \ttfamily \color{red!20!green!60!blue}{8(\%ebp)}}`
    pushl   `\underline{\mdseries \ttfamily \color{red!20!green!60!blue}{12(\%ebp)}}`
    pushl   `\underline{\mdseries \ttfamily \color{red!20!green!60!blue}{16(\%ebp)}}`
    pushl   `\underline{\mdseries \ttfamily \color{red!20!green!60!blue}{17(\%ebp)}}`
    call    f
    addl    $16, %esp
    leave
    ret
// \songti \upshape 当N=2时,补全后的汇编代码片段。
\end{verbatimwrite}

\begin{verbatimwrite}{Ex2_N=11_Filled.s}
    .file "test1.c"
    .text
.globl f
    .type   f,@function
f:
    pushl   %ebp
    movl    %esp, %ebp
    pushl   %edi
    pushl   %esi
    movl    $100, 8(%ebp)
    movl    `\underline{\mdseries \ttfamily \color{red!20!green!60!blue}{\$24}}`, 12(%ebp)
    movb    $65, `\underline{\mdseries \ttfamily \color{red!20!green!60!blue}{17(\%ebp)}}`
    subl    $8, %esp
    movl    `\underline{\mdseries \ttfamily \color{red!20!green!60!blue}{\$6}}`, %eax
    subl    $24, %esp
    movl    %esp, %edi
    movl    %eax, %esi
    cld
    movl    `\underline{\mdseries \ttfamily \color{red!20!green!60!blue}{\$34}}`, %eax
    movl    %eax, %ecx
    rep
    movsl
    call    f
    addl    $32, %esp
    leal    -8(%ebp), %esp
    popl    %esi
    popl    %edi
    leave
    ret
// \songti \upshape 当N=11时,补全后的汇编代码片段
\end{verbatimwrite}

\begin{verbatimwrite}{Ex3.c}
void g(int**);
int main()
{
    int line[10], i;
    int *p = line;
    for (i = 0; i < 10; i++)
    {
        *p = i;
        g(&p);
    }
    return 0;
}
void g(int**p)
{
    (**p)++;
    (*p)++;
}
// \songti \upshape 第三题C程序
\end{verbatimwrite}

\begin{verbatimwrite}{Ex3_g.s}
.globl g
    .type   g,@function
g:
    pushl   %ebp
    movl    %esp, %ebp
    movl    `\underline{~\textcircled{4}~}`, %eax
    movl    `\underline{~\textcircled{5}~}`, %eax
    `\underline{~~~~~~~~~\textcircled{6}~~~~~~~}`
    movl    `\underline{~\textcircled{7}~}`, %eax
    `\underline{~~~~~~~~~\textcircled{8}~~~~~~~}`
    leave
    ret
// \songti \upshape 第三题函数\texttt{g}的汇编代码片段
\end{verbatimwrite}

\begin{verbatimwrite}{Ex3_main.s}
    .file "p.c"
    .text
.globl main
    .type   main,@function
main:
    pushl   %ebp
    movl    %esp, %ebp
    subl    $72, %esp
    andl    $-16, %esp
    movl    $0, %eax
    subl    %eax, %esp
    leal    -56(%ebp), %eax
    movl    %eax, -64(%ebp)
    movl    $0, -60(%ebp)
.L2:
    `\underline{~~~~~~~~\textcircled{1}~~~~~~~~}`
    jle     .L5
    jmp     .L3
.L5:
    movl    -64(%ebp), %edx
    movl    -60(%ebp), %eax
    movl    %eax, (%edx)
    subl    $12, %esp
    leal    -64(%ebp), %eax
    pushl   %eax
    call    g
    `\underline{~~~~~~~~\textcircled{2}~~~~~~~~}`
    leal    -60(%ebp), %eax
    incl    (%eax)
    `\underline{~~~~~~~~\textcircled{3}~~~~~~~~}`
.L3:
    movl    $0, %eax
    leave
    ret
// \songti \upshape 第三题函数\texttt{main}的汇编代码片段
\end{verbatimwrite}

\begin{verbatimwrite}{Ex3_g_Filled.s}
.globl g
    .type   g,@function
g:
    pushl   %ebp
    movl    %esp, %ebp
    movl  `\underline{\textcircled{4}\ttfamily \color{red!20!green!60!blue}{8(\%ebp)}}`, %eax
    movl  `\underline{\textcircled{5}\ttfamily \color{red!20!green!60!blue}{(\%eax)}}`, %eax
  `\underline{\textcircled{6}\ttfamily \color{red!20!green!60!blue}{movl~~~~-1(\%eax), (\%eax)}}`
    movl  `\underline{\textcircled{7}\ttfamily \color{red!20!green!60!blue}{8(\%ebp)}}`, %eax
  `\underline{\textcircled{8}\ttfamily \color{red!20!green!60!blue}{movl~~~~(\%eax), \%eax}}`
    leave
    ret
// \songti \upshape 补全后第三题函数\texttt{g}的汇编代码片段
\end{verbatimwrite}

\begin{verbatimwrite}{Ex3_main_Filled.s}
    .file "p.c"
    .text
.globl main
    .type   main,@function
main:
    pushl   %ebp
    movl    %esp, %ebp
    subl    $72, %esp
    andl    $-16, %esp
    movl    $0, %eax
    subl    %eax, %esp
    leal    -56(%ebp), %eax
    movl    %eax, -64(%ebp)
    movl    $0, -60(%ebp)
.L2:
  `\underline{\textcircled{1}\ttfamily \color{red!20!green!60!blue}{cmpl~~~~\$9, -60(\%ebp)}}`
    jle     .L5
    jmp     .L3
.L5:
    movl    -64(%ebp), %edx
    movl    -60(%ebp), %eax
    movl    %eax, (%edx)
    subl    $12, %esp
    leal    -64(%ebp), %eax
    pushl   %eax
    call    g
  `\underline{\textcircled{2}\ttfamily \color{red!20!green!60!blue}{movl~~~~\%eax, (\%ebp)}}`
    leal    -60(%ebp), %eax
    incl    (%eax)
  `\underline{\textcircled{3}\ttfamily \color{red!20!green!60!blue}{jmpl~~~~.L2}}`
.L3:
    movl    $0, %eax
    leave
    ret
// \songti \upshape 补全后第三题函数\texttt{main}的汇编代码片段
\end{verbatimwrite}

\begin{verbatimwrite}{Ex4.c}
#include <stdio.h>
int main()
{
    int a = 0, b = 0;
    { int a = 1; }
    { int b = 2;
        { int a = 3; }
    }
    return 0;
}// \songti \upshape 第四题C程序
\end{verbatimwrite}

\begin{verbatimwrite}{Ex4_Unfilled.s}
main:
    pushl   %ebp
    `\underline{~~~~~~~~~~~~~~~~~~}`
    subl    $24, %esp
    andl    $-16, %esp
    movl    $0, %eax
    subl    %eax, %esp
    movl    $0, `\underline{~~~~~~~~~}`
    movl    $0, `\underline{~~~~~~~~~}`
    movl    $1, `\underline{~~~~~~~~~}`
    movl    $2, -12(%ebp)
    movl    $3, `\underline{~~~~~~~~~}`
    movl    $0, %eax
    leave
    ret
// \songti \upshape 第四题汇编代码片段
\end{verbatimwrite}

\begin{verbatimwrite}{Ex4_Filled.s}
main:
    pushl   %ebp
    `\underline{\ttfamily \color{red!20!green!60!blue}{movl~~~~\%esp, \%ebp}}`
    subl    $24, %esp
    andl    $-16, %esp
    movl    $0, %eax
    subl    %eax, %esp
    movl    $0, `\underline{\ttfamily \color{red!20!green!60!blue}{\%ebp}}`
    movl    $0, `\underline{\ttfamily \color{red!20!green!60!blue}{-4(\%ebp)}}`
    movl    $1, `\underline{\ttfamily \color{red!20!green!60!blue}{-8(\%ebp)}}`
    movl    $2, -12(%ebp)
    movl    $3, `\underline{\ttfamily \color{red!20!green!60!blue}{-16(\%ebp)}}`
    movl    $0, %eax
    leave
    ret
// \songti \upshape 补全后第四题汇编代码片段
\end{verbatimwrite}

\begin{verbatimwrite}{Ex5.c}
#include <stdio.h>
int main()
{
    int a[6] = {0, 1, 2, 3, 4, 5};
    int i = 6, j = 7;
    int *p = (int*)(&a + 1);
    printf("%d\n",*(p - 1));
    return 0;
} // \upshape \songti 第五题C程序
\end{verbatimwrite}

\begin{verbatimwrite}{Ex5_Unfilled.s}
.LC0:
    .long   0
    .long   1
    .long   2
    .long   3
    .long   4
    .long   5
.LC1:
    .string "%d\n"
    .text
.globl main
    .type   main, @function
main:
    pushl   %ebp
    movl    %esp, %ebp
    pushl   %edi
    pushl   %esi
    subl    $48, %esp
    andl    $-16, %esp
    movl    $0, %eax
    subl    %eax, %esp
`\uwave{~~~~leal~~~~-40(\%\color{blue}{ebp}), \%\color{blue}{edi}}`
`\uwave{~~~~movl~~~~\$.LC0, \%\color{blue}{esi}}`
`\uwave{~~~~\color{blue}{cld}}`
`\uwave{~~~~movl~~~~\$6, \%\color{blue}{eax}}`
`\uwave{~~~~movl~~~~\%\color{blue}{eax}, \%\color{blue}{ecx}}`
`\uwave{~~~~\color{blue}{rep}}`
`\uwave{~~~~movsl}`
    movl    $6, -44(%ebp)
    movl    $7, -48(%ebp)
    leal    -40(%ebp), %eax
    addl    `\underline{~~~~~~~~~~~~~~~}`
    movl    %eax, -52(%ebp)
    subl    $8, %esp
    movl    -52(%ebp), %eax
    subl    $`\underline{~~~~~~~~~}`, %eax
    pushl   `\underline{~~~~~~~~~~~~~~~}`
    pushl   $.LC1
    call    printf
    addl    $`\underline{~~~~~~~~~}` , %esp
    movl    $0, %eax
    leal    `\underline{~~~~~~~~~~}`, %esp
    popl    %esi
    popl    %edi
    leave
    ret
// \upshape \songti 第五题汇编代码片段
\end{verbatimwrite}

\begin{verbatimwrite}{Ex5_Filled.s}
.LC0:
    .long   0
    .long   1
    .long   2
    .long   3
    .long   4
    .long   5
.LC1:
    .string "%d\n"
    .text
.globl main
    .type   main, @function
main:
    pushl   %ebp
    movl    %esp, %ebp
    pushl   %edi
    pushl   %esi
    subl    $48, %esp
    andl    $-16, %esp
    movl    $0, %eax
    subl    %eax, %esp
    leal    -40(%ebp), %edi
    movl    $.LC0, %esi
    cld
    movl    $6, %eax
    movl    %eax, %ecx
    rep
    movsl
    movl    $6, -44(%ebp)
    movl    $7, -48(%ebp)
    leal    -40(%ebp), %eax
    addl    `\underline{\ttfamily \color{red!20!green!60!blue}{\$24, \%eax}}`
    movl    %eax, -52(%ebp)
    subl    $8, %esp
    movl    -52(%ebp), %eax
    subl    $`\underline{\ttfamily \color{red!20!green!60!blue}{4}}`, %eax
    pushl   `\underline{\ttfamily \color{red!20!green!60!blue}{(\%eax)}}`
    pushl   $.LC1
    call    printf
    addl    $`\underline{\ttfamily \color{red!20!green!60!blue}{8}}` , %esp
    movl    $0, %eax
    leal    `\underline{\ttfamily \color{red!20!green!60!blue}{-12(\%ebp)}}`, %esp
    popl    %esi
    popl    %edi
    leave
    ret
// \upshape \songti 补全后的第五题汇编代码片段
\end{verbatimwrite}

\begin{document}

\maketitle

\section*{Exercise 1}
针对如下C程序及其在i386 Linux下的汇编代码(片段):

\begin{multicols}{2}
\CodeBlock{C}{./Ex1.c}
\subparagraph*{(a)}
上述C程序的输出是什么?
\subparagraph*{(b)}
补全10处划线部分的汇编代码。
\CodeBlock{Assembler}{./Ex1_Unfilled.s}
\end{multicols}

\paragraph{解}
\subparagraph*{(a)}
程序的输出结果为
\begin{lstlisting}[
    language = sh,
    frame = single,
]
36313032 2016
\end{lstlisting}
\subparagraph*{(b)}
补全后的汇编代码如下
\CodeBlock{Assembler}{./Ex1_Filled.s}

\newpage

\section*{Exercise 2}
针对如下C程序及其汇编代码(片段):
% \begin{multicols}{2}
\subparagraph*{(a)}
补全划线处的汇编代码;
\subparagraph*{(b)}
从运行时环境看,\lstinline[style = Assembler]{addl $16, %esp}和\lstinline[style = Assembler]{leal -8(%ebp), %esp}这两条汇编指令的作用是什么?
\subparagraph*{(c)}
结合上述两种汇编代码,简述编译器在按值传递结构变量时的处理方式。

\begin{multicols}{2}
\CodeBlock{C}{./Ex2.c}

\CodeBlock{Assembler}{./Ex2_N=2.s}

\CodeBlock{Assembler}{./Ex2_N=11.s}
\end{multicols}

\vspace{5em}

\paragraph{解}
\subparagraph*{(a)}
补全后的汇编代码分别如下

\begin{multicols}{2}
\CodeBlock{Assembler}{./Ex2_N=2_Filled.s}
\CodeBlock{Assembler}{./Ex2_N=11_Filled.s}
\end{multicols}
\subparagraph*{(b)}
\lstinline[style = Assembler]{addl $16, %esp}指令将栈顶指针恢复到参数压栈之前的位置,\lstinline[style = Assembler]{leal -8(%ebp), %esp}加载栈顶指针地址为\lstinline[style = Assembler]{%ebp}的值减去8
\subparagraph{(c)}
编译器在按值传递结构变量时,会以处理局部变量的方式,为结构体成员变量分配空间。在递归调用中,完成实参的值压栈并调用递归函数后,将恢复栈顶指针到参数压栈前的位置。

\FloatBarrier

\newpage

\section*{Exercise 3}
针对如下C程序及其汇编代码(片段):

\begin{multicols}{2}
\CodeBlock{C}{./Ex3.c}

\CodeBlock{Assembler}{./Ex3_g.s}

\CodeBlock{Assembler}{./Ex3_main.s}
\end{multicols}

\subparagraph*{(a)}
补全下划线处的空白汇编代码;
\subparagraph*{(b)}
\texttt{main}函数中\texttt{for}循环结束时,数组\texttt{line}各元素值是多少?

\paragraph{解}
\subparagraph*{(a)}
补全后的汇编代码分别如下

\begin{multicols}{2}
\CodeBlock{Assembler}{./Ex3_g_Filled.s}
\subparagraph*{(b)}
\texttt{main}函数中\texttt{for}循环结束时,数组\texttt{line}各元素值如下
\begin{figure}[H]
\begin{lstlisting}[
    style = C,
    frame = single,
]
line[0] = 1
line[1] = 2
line[2] = 3
line[3] = 4
line[4] = 5
line[5] = 6
line[6] = 7
line[7] = 8
line[8] = 9
line[9] = 10
\end{lstlisting}
\end{figure}
\CodeBlock{Assembler}{./Ex3_main_Filled.s}
\end{multicols}




\section*{Exercise 4}
针对如下C程序及其汇编代码(片段):

\begin{multicols}{2}
\CodeBlock{C}{./Ex4.c}

\CodeBlock{Assembler}{./Ex4_Unfilled.s}
\end{multicols}

\subparagraph*{(a)}
补全下划线处的空白汇编代码;
\subparagraph*{(b)}
描述所用编译器对C分程序所声明变量的存储分配策略;

\paragraph{解}
\subparagraph*{(a)}
补全后的汇编代码如下

\CodeBlock{Assembler}{./Ex4_Filled.s}

\subparagraph*{(b)}
对于过程中出现的作用域不同的局部变量,编译器按出现次序为其在堆栈中分配空间。

\newpage

\begin{multicols}{2}
\section*{Exercise 5}
仔细阅读所给C程序及其汇编代码片段。

\subparagraph*{(a)}
指出\uwave{波浪线}处的汇编代码的作用;
\subparagraph*{(b)}
补全下划线处的空白汇编代码。

% \begin{multicols}{2}
\CodeBlock{C}{./Ex5.c}

\CodeBlock{Assembler}{./Ex5_Unfilled.s}
\end{multicols}

\paragraph{解}
\subparagraph*{(a)}
将从源地址\lstinline[style = Assembler]{$.LCO}开始的6个字的数据,即\lstinline{0, 1, 2, 3, 4, 5}传送到地址为\lstinline[style = Assembler]{%ebp - 40}即栈顶的存储单元中。
\subparagraph*{(b)}
补全后的汇编代码如下
\CodeBlock{Assembler}{./Ex5_Filled.s}

\section*{Exercise 6}
假设以下假想的程序采用静态嵌套作用域规则:
\begin{lstlisting}[
    language = Pascal,
    alsolanguage = C,
    frame = {}
]
program staticLink
    procedure f(level, arg())
    // \upshape \songti 函数f有两个参数,整型变量level,无参函数arg

        procedure local() // \upshape \songti 嵌套在f中的函数
        begin // \upshape \songti 无参函数local,返回一个整型值。
            return level;
        end

    begin // \upshape \songti f函数体
        if (level > 10) return f(level - 1, local);
        else if (level > 1) return f(level - 1, arg); else return arg();
    end

    procedure dummy()
    begin
        /* `\upshape \songti 空的函数体` */
    end

begin // \upshape \songti staticLink函数体
    print(f(17, dummy));
end
\end{lstlisting}
\subparagraph*{(a)}
给出该程序运行结果;
\subparagraph*{(b)}
给出函数调用\lstinline{f(17, dummy)}执行时运行栈上包含活动记录最多时的相关图示。假设按照逆序方式传递参数;函数作为参数传递时,需要两个单元,一为函数入口地址(可用函数名表示),二为其访问链。

\paragraph{解}
\subparagraph*{(a)}
程序的输出结果为
\begin{lstlisting}[
    language = sh,
    frame = single,
]
11
\end{lstlisting}
\subparagraph*{(b)}
\newgeometry{left = 0em}
\begin{figure}
    \centering
    \begin{tikzpicture}[node distance = 0em]
        \tikzstyle{block} = [rectangle, draw=black, text centered, minimum height = 1.6em, minimum width = 9em];
        \node [block, fill = green!00]                               (returned_value_0)  {返回值};
        \node [block, fill = green!21, below = of returned_value_0]  (returned_address_0){返回地址};
        \node [block, fill = green!28, below = of returned_address_0](caller_bp_0)       {调用者bp};
        \node [block, fill = blue!42, below = of caller_bp_0]       (temporaries_0_1)   {\texttt{f(level, arg())}};
        \node [block, fill = black!48, below = of temporaries_0_1]   (temporaries_0)     {\texttt{dummy()}};
        \node [block, fill = blue!00, below = of {temporaries_0}]     (returned_value_1)  {返回值};
        \node [block, fill = blue!07, below = of {returned_value_1}]  (parameters_1_0)    {\texttt{level = 17}};
        \node [block, fill = blue!07, below = of {parameters_1_0}]    (parameters_1)      {\texttt{arg = dummy}};
        \node [block, fill = blue!14, below = of {parameters_1}]      (link_1)            {\textsc{Point to} \texttt{dummy}};

        \node [block, fill = blue!21, below = of {link_1}]            (returned_address_1){返回地址};
        \node [block, fill = blue!28, below = of {returned_address_1}](caller_bp_1)       {调用者bp};

        \node [block, fill = yellow!40, below = of {caller_bp_1}]      (local_1)     {\texttt{local()}};
        \foreach \i/\j/\c/\level in {2/1/2/16, 3/2/2/15, 4/3/2/14, 5/4/4/13, 6/5/4/12, 7/6/4/11, 8/7/4/10, 9/8/6/9, 10/9/6/8, 11/10/6/7, 12/11/6/6, 13/12/8/5, 14/13/8/4, 15/14/8/3, 16/15/8/2, 17/16/8/1}
        {
            \ifnum \i = 4
                \node [block, fill = blue!00, right = of {returned_value_1}]     (returned_value_\i)  {返回值};
            \else
            \ifnum \i = 7
                \node [block, fill = blue!00, right = of {returned_value_4}]     (returned_value_\i)  {返回值};
            \else
            \ifnum \i = 10
                \node [block, fill = blue!00, right = of {returned_value_7}]     (returned_value_\i)  {返回值};
            \else
            \ifnum \i = 13
                \node [block, fill = blue!00, right = of {returned_value_10}]     (returned_value_\i)  {返回值};
            \else
            \ifnum \i = 16
                \node [block, fill = blue!00, right = of {returned_value_13}]     (returned_value_\i)  {返回值};
            \else
                \node [block, fill = blue!00, below = of {local_\j}]     (returned_value_\i)  {返回值};
            \fi
            \fi
            \fi
            \fi
            \fi
            \node [block, fill = blue!07, below = of {returned_value_\i}]  (parameters_\i_0)    {\texttt{level = \level}};
            \ifnum \i > 8
                \node [block, fill = blue!07, below = of {parameters_\i_0}]    (parameters_\i)      {\texttt{arg = arg}};
            \else
                \node [block, fill = blue!07, below = of {parameters_\i_0}]    (parameters_\i)      {\texttt{arg = local}};
            \fi
            \node [block, fill = blue!14, below = of {parameters_\i}]      (link_\i)            {\textsc{Point to} \texttt{local}};

            \node [block, fill = blue!21, below = of {link_\i}]            (returned_address_\i){返回地址};
            \node [block, fill = blue!28, below = of {returned_address_\i}](caller_bp_\i)       {调用者bp};

            \node [block, fill = yellow!40, below = of {caller_bp_\i}]      (local_\i)     {\texttt{local()}};
        }
        \node [block, fill = yellow!00, below = of {local_17}]     (returned_value)  {返回值 = 11};
        \node [block, fill = yellow!14, below = of {returned_value}]            (returned_address){返回地址};
        \node [block, fill = yellow!28, below = of {returned_address}](caller_bp)       {调用者bp};


        \draw [-triangle 60] (link_1) to [out = 150, in = -150] (temporaries_0);
        \draw [-open triangle 45, thick, green!60!black] (caller_bp_1) to [out = 150, in = -150] (caller_bp_0);
        \draw [o-stealth, thick, red!60!black] (returned_value_1) to [out = 30, in = -30] (returned_value_0);
        \foreach \i/\j/\c/\level in {2/1/2/16, 3/2/2/15, 4/3/2/14, 5/4/4/13, 6/5/4/12, 7/6/4/11, 8/7/4/10, 9/8/6/9, 10/9/6/8, 11/10/6/7, 12/11/6/6, 13/12/8/5, 14/13/8/4, 15/14/8/3, 16/15/8/2, 17/16/8/1}
        {
            \ifnum \i > 8
                \draw [-triangle 60, thick] (link_\i) to [] (local_7);
                \ifnum \i = 10
                    \draw [-open triangle 45, thick, green!60!black] (caller_bp_\i) to [out = -135, in = 45] (caller_bp_\j);
                    \draw [o-stealth, thick, red!60!black] (returned_value_\i) to [out = -90, in = 90] (returned_value_\j);
                \else
                \ifnum \i = 13
                    \draw [-open triangle 45, thick, green!60!black] (caller_bp_\i) to [out = -135, in = 45] (caller_bp_\j);
                    \draw [o-stealth, thick, red!60!black] (returned_value_\i) to [out = -90, in = 90] (returned_value_\j);
                \else
                \ifnum \i = 16
                    \draw [-open triangle 45, thick, green!60!black] (caller_bp_\i) to [out = -135, in = 45] (caller_bp_\j);
                    \draw [o-stealth, thick, red!60!black] (returned_value_\i) to [out = -90, in = 90] (returned_value_\j);
                \else
                    \draw [-open triangle 45, thick, green!60!black] (caller_bp_\i) to [out = 150, in = -150] (caller_bp_\j);
                    \draw [o-stealth, thick, red!60!black] (returned_value_\i) to [out = 30, in = -30] (returned_value_\j);
                \fi
                \fi
                \fi
            \else
            \ifnum \i = 4
                \draw [-triangle 60] (link_\i) to [out = -90, in = 90] (local_\j);
                \draw [-open triangle 45, thick, green!60!black] (caller_bp_\i) to [out = -135, in = 45] (caller_bp_\j);
                \draw [o-stealth, thick, red!60!black] (returned_value_\i) to [out = -90, in = 90] (returned_value_\j);
            \else
            \ifnum \i = 7
                \draw [-triangle 60] (link_\i) to [out = -90, in = 90] (local_\j);
                \draw [-open triangle 45, thick, green!60!black] (caller_bp_\i) to [out = -135, in = 45] (caller_bp_\j);
                \draw [o-stealth, thick, red!60!black] (returned_value_\i) to [out = -90, in = 90] (returned_value_\j);
            \else
                \draw [-triangle 60] (link_\i) to [out = 150, in = -150] (local_\j);
                \draw [-open triangle 45, thick, green!60!black] (caller_bp_\i) to [out = 150, in = -150] (caller_bp_\j);
                \draw [o-stealth, thick, red!60!black] (returned_value_\i) to [out = 30, in = -30] (returned_value_\j);

            \fi
            \fi
            \fi
        }
        \draw [-open triangle 45, thick, green!60!black] (caller_bp) to [out = 0, in = 0] (caller_bp_17);
        \draw [o-stealth, thick, red!60!black] (returned_value) to [out = 30, in = -30] (returned_value_17);
    \end{tikzpicture}
    \caption{栈上包含19个活动记录表时的图示}
\end{figure}
\end{document}

