\documentclass{article}
\usepackage[UTF8]{ctex}
\usepackage[T1]{fontenc}
\usepackage[utf8]{inputenc}
\usepackage{float}
\usepackage{placeins}
\usepackage{latexsym}
\usepackage{amsmath}
\usepackage{amsthm}
\usepackage{listings}
\usepackage{xcolor}
\usepackage{multicol}
\usepackage{geometry}
\usepackage{mdframed}
\usepackage{moreverb}
\lstset
{
    basicstyle=\ttfamily,
    keywordstyle=\bfseries\color{blue!70},
    frame = single,
    escapeinside=``,
    breaklines = true,
    breakatwhitespace = true,
    breakautoindent = true,
    texcl = true,
    showstringspaces = false,
    flexiblecolumns,
    columns = fixed,
    basewidth = 0.5em,
}
\lstdefinestyle{C}
{
    language = C,
    basicstyle=\ttfamily,
    keywordstyle=\bfseries\color{blue!70},
}
\lstdefinestyle{Assembler}
{
    language = [X86masm]Assembler,
    basicstyle=\ttfamily,
    keywordstyle=\bfseries\color{blue!70},
}
\mdfdefinestyle{Default}
{
    innertopmargin = 0pt,
    innerbottommargin = 0pt,
    innerleftmargin = 3pt,
    innerrightmargin = 3pt,
    linewidth = 0.5pt,
}

\title{Homework 6}
\author{PB17000297 罗晏宸}
\date{October 23 2019}
\begin{document}
\maketitle

\section*{Exercise 1}
针对如下C程序及其在i386 Linux下的汇编代码(片段):
\begin{figure}
    \begin{multicols}{2}
        \begin{figure}[H]
            \centering
            \begin{lstlisting}[
                style = C,
                xleftmargin = 27pt,
            ]
#include<stdio.h>
union var{
    char c[5];
    int i;
};
int main(){
    union var data;
    char *c;
    data.c[0] = '2';
    data.c[1] = '0';
    data.c[2] = '1';
    data.c[3] = '6';
    data.c[4] = '\0';
    c = (char*)&data;
    printf("%x %s\n", data.i, c);
    return 0;
}
// \songti \upshape 第一题C程序
        \end{lstlisting}
        \end{figure}
        \begin{figure}[H]
            \centering
            \begin{lstlisting}[
                style = Assembler,
                alsolanguage = C,
                xrightmargin = 60pt
            ]
    .section .rodata
.LC0:
    .string "%x %s\n"
    .text
.globl main
    .type   main, @function
main:
    `\underline{~~~~~~~~~~~~~~~~~~~~~~}`
    movl    %esp, %ebp
    subl    $40, %esp
    andl    $-16, %esp
    movl    $0, %eax
    subl    %eax, %esp
    movb    $50, -24(%ebp)
    `\underline{~~~~~~~~~~~~~~~~~~~~~~}`
    `\underline{~~~~~~~~~~~~~~~~~~~~~~}`
    `\underline{~~~~~~~~~~~~~~~~~~~~~~}`
    `\underline{~~~~~~~~~~~~~~~~~~~~~~}`
    `\underline{~~~~~~~~~~~~~~~~~~~~~~}`
    movl    %eax, -28(%ebp)
    `\underline{~~~~~~~~~~~~~~~~~~~~~~}`
    `\underline{~~~~~~~~~~~~~~~~~~~~~~}`
    `\underline{~~~~~~~~~~~~~~~~~~~~~~}`
    pushl   $.LC0
    call    printf
    addl    $16, %esp
    `\underline{~~~~~~~~~~~~~~~~~~~~~~}`
    leave
    ret
// \songti 第一题 汇编程序
            \end{lstlisting}
        \end{figure}
    \end{multicols}
\end{figure}
\subsection*{(a)}
上述C程序的输出是什么?
\subsection*{(b)}
补全10处划线部分的汇编代码。

\paragraph{解}
\subparagraph*{(a)}
程序的输出结果为
\begin{lstlisting}[style = C]
36313032 2016
\end{lstlisting}
\subparagraph*{(b)}
补全后的汇编代码如下
\begin{figure}[H]
    \centering
    \begin{lstlisting}[
        style = Assembler,
        alsolanguage = C,
    ]
    .section .rodata
.LC0:
    .string "%x %s\n"
    .text
.globl main
    .type   main, @function
main:
    `\underline{\mdseries \ttfamily \color{red!20!green!60!blue}{pushl~~~\%ebp}}`
    movl    %esp, %ebp
    subl    $40, %esp
    andl    $-16, %esp
    movl    $0, %eax
    subl    %eax, %esp
    movb    $50, -24(%ebp)
    `\underline{\mdseries \ttfamily \color{red!20!green!60!blue}{movb~~~~\$48, -23(\%ebp)}}`
    `\underline{\mdseries \ttfamily \color{red!20!green!60!blue}{movb~~~~\$49, -22(\%ebp)}}`
    `\underline{\mdseries \ttfamily \color{red!20!green!60!blue}{movb~~~~\$54, -21(\%ebp)}}`
    `\underline{\mdseries \ttfamily \color{red!20!green!60!blue}{movb~~~~\$0, -20(\%ebp)}}`
    `\underline{\mdseries \ttfamily \color{red!20!green!60!blue}{leal~~~~-24(\%ebp), \%eax}}`
    movl    %eax, -28(%ebp)
    `\underline{\mdseries \ttfamily \color{red!20!green!60!blue}{movl~~~~-24(\%ebp), \%eax}}`
    `\underline{\mdseries \ttfamily \color{red!20!green!60!blue}{movl~~~~-28(\%ebp), \%edx}}`
    `\underline{\mdseries \ttfamily \color{red!20!green!60!blue}{movl~~~~\%edx, \%esp}}`
    pushl   $.LC0
    call    printf
    addl    $16, %esp
    `\underline{\mdseries \ttfamily \color{red!20!green!60!blue}{popl~~~~\%esp}}`
    leave
    ret
    \end{lstlisting}
\end{figure}

\section*{Exercise 2}
针对如下C程序及其汇编代码(片段):
\begin{figure}
    \begin{multicols}{2}
        \begin{figure}[H]
            \centering
            \begin{verbatimwrite}{Ex2.c}
#define N 2
// \upshape \ttfamily \#define N 11
typedef struct POINT {
    int x, y ;
    char z[ N ];
    struct POINT *next;
} DOT;
void f(DOT p)
{
    p.x = 100;
    p.y = sizeof(p);
    p.z[1] = 'A';
    f(*(p.next));
}
// \songti \upshape 第二题C程序
            \end{verbatimwrite}
            \begin{mdframed}[
                style = Default,
                leftmargin = 74pt,
            ]
                \lstinputlisting[
                    style = C,
                    frame = {},
                    aboveskip = 0pt,
                ]{./Ex2.c}
            \end{mdframed}
        \end{figure}
        \begin{figure}[H]
            \centering
            \begin{verbatimwrite}{Ex2_N=2.s}
    .file "test1.c"
    .text
.globl f
    .type   f,@function
f:
    pushl   %ebp
    movl    %esp, %ebp
    movl    $100, 8(%ebp)
    movl    $16, 12(%ebp)
    movb    $65, `\underline{~~~~~~~~}`
    movl    `\underline{~~~~~~~}`, %eax
    pushl   `\underline{~~~~~~~~~~~~~}`
    pushl   `\underline{~~~~~~~~~~~~~}`
    pushl   `\underline{~~~~~~~~~~~~~}`
    pushl   `\underline{~~~~~~~~~~~~~}`
    call    f
    addl    $16, %esp
    leave
    ret
// \songti \upshape 当N=2时,生成的汇编代码片段。
            \end{verbatimwrite}
            \begin{mdframed}[
                style = Default,
                leftmargin = 30pt,
            ]
                \lstinputlisting[
                    style = Assembler,
                    alsolanguage = C,
                    frame = {},
                    aboveskip = 0pt,
                ]{./Ex2_N=2.s}
            \end{mdframed}
        \end{figure}
        \begin{figure}[H]
            \centering
            \begin{verbatimwrite}{Ex2_N=11.s}
    .file "test1.c"
    .text
.globl f
    .type   f,@function
f:
    pushl   %ebp
    movl    %esp, %ebp
    pushl   %edi
    pushl   %esi
    movl    $100, 8(%ebp)
    movl    `\underline{~~~}`, 12(%ebp)
    movb    $65, `\underline{~~~~~~~}`
    subl    $8, %esp
    movl    `\underline{~~~~~~}`, %eax
    subl    $24, %esp
    movl    %esp, %edi
    movl    %eax, %esi
    cld
    movl    `\underline{~~~~~~}`, %eax
    movl    %eax, %ecx
    rep
    movsl
    call    f
    addl    $32, %esp
    leal    -8(%ebp), %esp
    popl    %esi
    popl    %edi
    leave
    ret
// \songti \upshape rep movsl为数据传送指令,即,由源地址esi开始的ecx个字的数据传送到由edi指示的目的地址。
// \songti \upshape 当N=11时,生成的汇编代码片段
            \end{verbatimwrite}
            \begin{mdframed}[
                style = Default,
                rightmargin = 25pt,
            ]
                \lstinputlisting[
                    style = Assembler,
                    alsolanguage = C,
                    frame = {},
                    aboveskip = 0pt,
                ]{./Ex2_N=11.s}
            \end{mdframed}
        \end{figure}
    \end{multicols}
\end{figure}
\subsection*{(a)}
补全划线处的汇编代码;
\subsection*{(b)}
从运行时环境看,\lstinline[style = Assembler]{addl $16, %esp}和\lstinline[style = Assembler]{leal -8(%ebp), %esp}这两条汇编指令的作用是什么?
\subsection*{(c)}
结合上述两种汇编代码,简述编译器在按值传递结构变量时的处理方式。

\paragraph{解}
\subparagraph*{(a)}
补全后的汇编代码分别如下
\begin{figure}[H]
    \centering
    \begin{multicols}{2}
        \begin{figure}[H]
            \centering
            \begin{verbatimwrite}{Ex2_N=2_Filled.s}
    .file "test1.c"
    .text
.globl f
    .type   f,@function
f:
    pushl   %ebp
    movl    %esp, %ebp
    movl    $100, 8(%ebp)
    movl    $16, 12(%ebp)
    movb    $65, `\underline{\mdseries \ttfamily \color{red!20!green!60!blue}{17(\%ebp)}}`
    movl    `\underline{\mdseries \ttfamily \color{red!20!green!60!blue}{20(\%ebp)}}`, %eax
    pushl   `\underline{\mdseries \ttfamily \color{red!20!green!60!blue}{8(\%ebp)}}`
    pushl   `\underline{\mdseries \ttfamily \color{red!20!green!60!blue}{12(\%ebp)}}`
    pushl   `\underline{\mdseries \ttfamily \color{red!20!green!60!blue}{16(\%ebp)}}`
    pushl   `\underline{\mdseries \ttfamily \color{red!20!green!60!blue}{17(\%ebp)}}`
    call    f
    addl    $16, %esp
    leave
    ret
// \songti \upshape 当N=2时,补全后的汇编代码片段。
            \end{verbatimwrite}
            \begin{mdframed}[
                style = Default,
                leftmargin = 20pt,
            ]
                \lstinputlisting[
                    style = Assembler,
                    alsolanguage = C,
                    frame = {},
                    aboveskip = 0pt,
                ]{./Ex2_N=2_Filled.s}
            \end{mdframed}
        \end{figure}
        \begin{figure}[H]
            \centering
            \begin{verbatimwrite}{Ex2_N=11_Filled.s}
    .file "test1.c"
    .text
.globl f
    .type   f,@function
f:
    pushl   %ebp
    movl    %esp, %ebp
    pushl   %edi
    pushl   %esi
    movl    $100, 8(%ebp)
    movl    `\underline{\mdseries \ttfamily \color{red!20!green!60!blue}{\$24}}`, 12(%ebp)
    movb    $65, `\underline{\mdseries \ttfamily \color{red!20!green!60!blue}{17(\%ebp)}}`
    subl    $8, %esp
    movl    `\underline{\mdseries \ttfamily \color{red!20!green!60!blue}{\$6}}`, %eax
    subl    $24, %esp
    movl    %esp, %edi
    movl    %eax, %esi
    cld
    movl    `\underline{\mdseries \ttfamily \color{red!20!green!60!blue}{\$34}}`, %eax
    movl    %eax, %ecx
    rep
    movsl
    call    f
    addl    $32, %esp
    leal    -8(%ebp), %esp
    popl    %esi
    popl    %edi
    leave
    ret
// \songti \upshape 当N=11时,补全后的汇编代码片段
            \end{verbatimwrite}
            \begin{mdframed}[
                style = Default,
                rightmargin = 15pt,
            ]
                \lstinputlisting[
                    style = Assembler,
                    alsolanguage = C,
                    frame = {},
                    aboveskip = 0pt,
                ]{./Ex2_N=11_Filled.s}
            \end{mdframed}
        \end{figure}
    \end{multicols}
\end{figure}
\subparagraph*{(b)}
\lstinline[style = Assembler]{addl $16, %esp}指令将栈顶指针恢复到参数压栈之前的位置,\lstinline[style = Assembler]{leal -8(%ebp), %esp}加载栈顶指针地址为\lstinline[style = Assembler]{%ebp}的值减去8
\subparagraph{(c)}
编译器在按值传递结构变量时,会以处理局部变量的方式,为结构体成员变量分配空间。在递归调用中,完成实参的值压栈并调用递归函数后,将恢复栈顶指针到参数压栈前的位置。

\section*{Exercise 3}
针对如下C程序及其汇编代码(片段):
\begin{figure}
    \begin{multicols}{2}
        \begin{figure}[H]
            \centering
            \begin{verbatimwrite}{Ex3.c}
void g(int**);
int main()
{
    int line[10], i;
    int *p = line;
    for (i = 0; i < 10; i++)
    {
        *p = i;
        g(&p);
    }
    return 0;
}
void g(int**p)
{
    (**p)++;
    (*p)++;
}
// \songti \upshape 第三题C程序
        \end{verbatimwrite}
        \begin{mdframed}[
            style = Default,
            leftmargin = 40pt,
        ]
            \lstinputlisting[
                style = C,
                frame = {},
                aboveskip = 0pt,
            ]{./Ex3.c}
        \end{mdframed}
        \end{figure}
        \begin{figure}[H]
            \centering
            \begin{verbatimwrite}{Ex3_g.s}
.globl g
    .type   g,@function
g:
    pushl   %ebp
    movl    %esp, %ebp
    movl    `\underline{~\textcircled{4}~}`, %eax
    movl    `\underline{~\textcircled{5}~}`, %eax
    `\underline{~~~~~~~~~\textcircled{6}~~~~~~~}`
    movl    `\underline{~\textcircled{7}~}`, %eax
    `\underline{~~~~~~~~~\textcircled{8}~~~~~~~}`
    leave
    ret
// \songti \upshape 第三题函数\texttt{g}的汇编代码片段
            \end{verbatimwrite}
            \begin{mdframed}[
                style = Default,
                leftmargin = 40pt,
            ]
                \lstinputlisting[
                    style = Assembler,
                    alsolanguage = C,
                    frame = {},
                    aboveskip = 0pt,
                ]{./Ex3_g.s}
            \end{mdframed}
        \end{figure}
        \begin{figure}[H]
            \centering
            \begin{verbatimwrite}{Ex3_main.s}
    .file "p.c"
    .text
.globl main
    .type   main,@function
main:
    pushl   %ebp
    movl    %esp, %ebp
    subl    $72, %esp
    andl    $-16, %esp
    movl    $0, %eax
    subl    %eax, %esp
    leal    -56(%ebp), %eax
    movl    %eax, -64(%ebp)
    movl    $0, -60(%ebp)
.L2:
    `\underline{~~~~~~~~\textcircled{1}~~~~~~~~}`
    jle     .L5
    jmp     .L3
.L5:
    movl    -64(%ebp), %edx
    movl    -60(%ebp), %eax
    movl    %eax, (%edx)
    subl    $12, %esp
    leal    -64(%ebp), %eax
    pushl   %eax
    call    g
    `\underline{~~~~~~~~\textcircled{2}~~~~~~~~}`
    leal    -60(%ebp), %eax
    incl    (%eax)
    `\underline{~~~~~~~~\textcircled{3}~~~~~~~~}`
.L3:
    movl    $0, %eax
    leave
    ret
// \songti \upshape 第三题函数\texttt{main}的汇编代码片段
            \end{verbatimwrite}
            \begin{mdframed}[
                style = Default,
                rightmargin = 25pt,
            ]
                \lstinputlisting[
                    style = Assembler,
                    alsolanguage = C,
                    frame = {},
                    aboveskip = 0pt,
                ]{./Ex3_main.s}
            \end{mdframed}
        \end{figure}
    \end{multicols}
\end{figure}
\subsection*{(a)}
补全划线处的汇编代码;
\subsection*{(b)}
从运行时环境看,\lstinline[style = Assembler]{addl $16, %esp}和\lstinline[style = Assembler]{leal -8(%ebp), %esp}这两条汇编指令的作用是什么?

\end{document}
