\documentclass{article}
\usepackage[UTF8]{ctex}
\usepackage[T1]{fontenc}
\usepackage[utf8]{inputenc}
\usepackage{float}
\usepackage{placeins}
\usepackage{latexsym}
\usepackage{amsmath}
\usepackage{amsthm}
\usepackage{listings}
\usepackage{xcolor}
\usepackage{ulem}
\lstset
{
    basicstyle = \ttfamily,
    keywordstyle = \bfseries\color{blue!70},
    commentstyle = \songti \upshape,
    escapeinside=``,
    breaklines = true,
    breakatwhitespace = true,
    breakautoindent = true,
    texcl = true,
    showstringspaces = false,
    flexiblecolumns,
    columns = fixed,
    frame = {},
}
\lstdefinestyle{C}
{
    language = C,
}
\lstdefinestyle{C++}
{
    language = C++,
}

\title{Homework 7\\ \normalsize PL/0 Exercise1}
\author{PB17000297 罗晏宸}
\date{October 29 2019}


\begin{document}

\maketitle

\textbf{阅读PL/0编译器相关文档,完成以下任务:}\par
在PL/0编译器中, 函数\lstinline{interpret()}在解释指令\texttt{LOD}/\texttt{STO}时的语义代码如下:

\begin{figure}[H]
    \centering
    \begin{lstlisting}[style = C]
case LOD: // 指令格式 (LOD, l, a)
    stack[++top] = stack[base(stack, b, i.l) + i.a];
    break;
case STO: // 指令格式 (STO, l, a)
    stack[base(stack, b, i.l) + i.a] = stack[top];
    printf("%d\n", stack[top]);
    top--;
    break;
    \end{lstlisting}
\end{figure}

\paragraph{(a)}
你“扩展”PL/0 编译器, 添加了 \texttt{LEA}/\texttt{LODA}/\texttt{STOA} 等指令。格式为: \lstinline{(LEA, l, a)}, \lstinline{(LODA, 0, 0)} 和 \lstinline{(STOA, 0, 0)}。 其中“取地址”指令 \texttt{LEA} 用来获取名字变量在“运行时栈-stack”上“地址偏移”。 而“间接读”指令 \texttt{LODA} 则表示以当前栈顶单元的内容为“地址偏移”来读取相应单元的值,并将该值存储到原先的栈顶单元中。而“间接写”指令 \texttt{STOA} 则将位于栈顶单元的内容,存入到次栈顶单元内容所代表的栈单元里, 然后弹出栈顶和次栈顶。\uline{给出这样的 \texttt{LODA}/\texttt{STOA}/\texttt{LEA} 指令的语义代码。}

\paragraph{解}
语义代码如下

\begin{figure}[H]
    \centering
    \begin{lstlisting}[style = C]
case LODA: // 指令格式 (LODA, 0, 0)
    stack[++top] = stack[stack[base(stack, b, i.l) + i.a]];
    break;
case STOA: // 指令格式 (STOA, 0, 0)
    stack[stack[top - 1]] = stack[top];
    printf("%d\n", stack[top]);
    top = top - 2;
    break;
case LEA: // 指令格式 (LEA, l, a)
    stack[++top] = base(stack, b, i,l) + i.a;
    break;
    \end{lstlisting}
\end{figure}

\paragraph{(b)}
现在继续扩展 PL/0 编译器。假设你实现了若干 C 风格的表达式、 类型及其声明体系,并可编译如下程序:

\begin{figure}[H]
    \centering
\begin{lstlisting}[style = C++, frame = single]
int main()
{
int  i;
int* q;
int* a[10];
int* (*b[10])[10];
int* (*(*p)[10])[10];

i = 100; q = &i; a[1] = q; b[1] = &a; p = &b;

cout <<`\uline{~~~~~\texttt{p[0]}~~~~~}`<< endl; //  输出100,待补全

cout <<`\uline{~~~~~~\texttt{*p}~~~~~~}`<< endl; // 输出100,待补全

} // 程序
\end{lstlisting}
\end{figure}

\begin{itemize}
    \item \uline{给出变量\texttt{a}和\texttt{p}的类型表达式。}\\
    注意, \lstinline{int} 即为类型表达式。 指针类型表示为 \lstinline{pointer(T1)}, \lstinline{T1} 为指针所指向对象的类型表达式,数组类型表示为 \lstinline{array(number,T2)}, 数组元素的个数 \lstinline{number} 为常量值, \lstinline{T2} 为数组元素的类型表达式。
    \item \uline{根据 PL/0 编译环境设定,上述程序中分配的总变量空间是多少? 各个变量在活动记录中“地址偏移”是多少?}
    \item \uline{两条输出语句中不同的表达式各自仅包含唯一的名字变量 \texttt{p}。根据你的 C 语言知识,补全这两处输出语句中的源代码。}
    \item \uline{给出一个上述下划线处源代码对应的 PL/0 代码} (两处输出语句可任选其一产生 PL/0 代码)。 如需使用算术运算,可直接给出,例如, 加法\lstinline{(ADD, 0, 0)}, 乘法\lstinline{(MUL, 0, 0)},以及加载常数于栈顶 \lstinline{(LIT, 0, 100)}等指令。
\end{itemize}

\paragraph{解}
\begin{itemize}
    \item 变量\texttt{a}是一个含10个元素的指针数组,类型表达式为\\
    \centerline{\lstinline{array(10, pointer(int))}}\\
    变量\texttt{p}是一个指针,指向一个含10个元素的数组,数组的每个元素是一个指向含10个元素数组的指针,被指向数组的每个元素是一个指向整型变量的指针,类型表达式为\\
    \centerline{\lstinline{pointer(array(10, pointer(array(10, (pointer(int))))))}}。
    \item 为变量\texttt{i}分配1字节,变量\texttt{q}分配1字节,变量\texttt{a}分配$1 \times 10 = 10$字节,变量\texttt{b}分配$1 \times 10 = 10$字节,变量\texttt{p}分配1字节,故分配的总变量空间为$1 + 1 + 10 + 10 + 1 = 23$字节;\\
    各个变量在活动记录中“地址偏移”分别是变量\texttt{i}偏移0字节,变量\texttt{q}偏移$0 + 1 = 1$字节,变量\texttt{a}偏移$1 + 1 = 2$字节,变量\texttt{b}偏移$2 + 10 = 12$字节,变量\texttt{p}偏移$12 + 10 = 22$字节。
    \item 代码补全如下
    \begin{figure}[H]
        \begin{lstlisting}[style = C++, frame = single]
cout <<`\underline{\texttt{{\color{red!20!green!60!blue}{*(*}}p[0]{\color{red!20!green!60!blue}{[1])[1]}}}}`<< endl; // 输出100
cout <<`\underline{\texttt{{\color{red!20!green!60!blue}{**(*(}}*p{\color{red!20!green!60!blue}{ + 1) + 1)}}}}`<< endl; // 输出100
        \end{lstlisting}
    \end{figure}
    \item 对应的 PL/0 代码如下
    \begin{figure}[H]
    \begin{lstlisting}[language =  PL/I, alsolanguage = C, frame = single]
(LIT, 0, 100)
(STO, 0, 0)
(LEA, 0, 2)
(LIT, 0, 1)
(ADD, 0, 0)
(LOD, 0, 1)
(STOA, 0, 0)
(LEA, 0, 22)
(LEA, 0, 12)
(STOA, 0, 0)
(LEA, 0, 22)
(LODA, 0, 0)
(LIT, 0, 1)
(ADD, 0, 0)
(LODA, 0, 0)
(LODA, 0, 0)
(LIT, 0, 1)
(ADD, 0, 0)
(LODA, 0, 0)
    \end{lstlisting}
\end{figure}
\end{itemize}

\paragraph{(c)}
再扩展 PL/0 编译器,你添加了“引用”声明及处理。一个引用变量也具有一个地址单元,其中存储着被“引用”的其他变量的地址偏移。\\
\uline{考虑如下程序片段:}
\begin{figure}[H]
    \centering
\begin{lstlisting}[style = C]
int* &r = `…` // \texttt{r} 是一个引用变量, 被引用对象是一个 \texttt{int} 指针变量。
int func(int *i, int* &j, int k); // 函数 \texttt{func} 声明
\end{lstlisting}
\end{figure}
\uline{对于函数调用: \lstinline{func(r, r, *r)} 分别给出计算三个实参的“值”到 stack 栈顶的
PL/0 代码。 假设\texttt{r}的地址偏移为3。}

\paragraph{解}
PL/0 代码如下
\begin{figure}[H]
    \begin{lstlisting}[language =  PL/I, alsolanguage = C, frame = single]
(LOD, 0, 3)
(LOD, 0, 3)
(LEA, 0, 3)
    \end{lstlisting}
\end{figure}
\end{document}

