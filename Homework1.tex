\documentclass{article}
\usepackage[UTF8]{ctex}
\usepackage[T1]{fontenc}
\usepackage[utf8]{inputenc}
\usepackage{float}
\usepackage{placeins}
\usepackage{latexsym}
\usepackage{amsmath}
\usepackage{amsthm}
\usepackage{listings}

\title{Homework 1}
\author{PB17000297 罗晏宸}
\date{September 1 2019}

\begin{document}
\maketitle

\section{Exercise 1}
观察讲义lec1中P4和P11上的函数fact的C代码及其汇编代码,初步了解编译器的作用。你可以:\par
(a)简要注释每条汇编代码;\par
(b)尝试指出C程序与汇编代码间的联系,\par
比如,C程序中的参数n在汇编中是如何表示的;if语句对应哪几条汇编代码......
\\

\paragraph{解}
fact是一个递归调用自身的函数,对应到汇编代码中的栈。汇编代码注释如下
\begin{lstlisting}[language={[x86masm]Assembler},numbers=left, 
         numberstyle=\tiny,keywordstyle=\color{blue!70},escapeinside=··,frame=shadowbox,
         rulesepcolor=\color{red!20!green!20!blue!20},basicstyle=\ttfamily]
fact:
	pushl	%ebp ·//将基址指针寄存器压栈,即保存$n$值·
	movl	%esp, %ebp
	subl	$4, %esp
	cmpl	$0, 8(%ebp)
	jg	.L2
	movl	$1, -4(%ebp)
	jmp	.L1
.L2:
	subl	$12, %esp
	movl	8(%ebp), %eax
	decl	%eax
	pushl	%eax
	call	fact
	addl	$16, %esp
	imull	8(%ebp), %eax
	movl	%eax, -4(%ebp)
.L1:
	movl	-4(%ebp), %eax
	leave
	ret

\end{lstlisting}

\end{document}
