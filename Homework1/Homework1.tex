\documentclass{article}
\usepackage[UTF8]{ctex}
\usepackage[T1]{fontenc}
\usepackage[utf8]{inputenc}
\usepackage{float}
\usepackage{placeins}
\usepackage{latexsym}
\usepackage{amsmath}
\usepackage{amsthm}
\usepackage{listings}
\usepackage{xcolor}


\title{Homework 1}
\author{PB17000297 罗晏宸}
\date{September 2 2019}

\begin{document}
\maketitle

\section{Exercise 1}
观察讲义lec1中P4和P11上的函数\texttt{fact}的C代码及其汇编代码,初步了解编译器的作用。你可以:\par
(a)简要注释每条汇编代码;\par
(b)尝试指出C程序与汇编代码间的联系,\par
比如,C程序中的参数\texttt{n}在汇编中是如何表示的;\texttt{if}语句对应哪几条汇编代码......
\\

\paragraph{解}
\texttt{fact}是一个递归调用自身的函数,对应到汇编代码中的栈。汇编代码注释如下
\begin{lstlisting}[language={[X86masm]Assembler}, 
         numbers=left, 
         numberstyle=\tiny,
         keywordstyle=\bfseries\color{blue!70},
         commentstyle=\small\color{red!40!green!60!blue},
         basicstyle=\small\ttfamily,
         escapeinside=``,
         breaklines=false,
         frame=lines]
fact:
	pushl	%ebp           ;将基址指针寄存器压栈,即保存n值
	movl	%esp, %ebp     ;将堆栈指针寄存器传送给基址指针寄存器,作为函数帧
	subl	$4, %esp       ;将栈指针下移4
	cmpl	$0, 8(%ebp)    ;从地址%ebp+8中取数与0比较:n-0(if语句条件)
	jg	.L2            ;大于(即n-0>0)则跳转至.L2标记
	movl	$1, -4(%ebp)   ;向地址%ebp-4中传入1(未跳转则n<=0,返回1)
	jmp	.L1            ;无条件跳转至.L1标记(以上实现if语句)
.L2:
	subl	$12, %esp      ;将栈指针下移12
	movl	8(%ebp), %eax  ;向%eax中传入地址%ebp+8中的数据,即返回值为n
	decl	%eax           ;%eax中的值自减1,即n-1
	pushl	%eax           ;将%ebx压栈,作为递归调用的参数
	call	fact           ;调用fact(n-1)
	addl	$16, %esp      ;将栈指针上移16(即回到基址)
	imull	8(%ebp), %eax  ;将%eax与地址%ebp+8中的数据相乘,即n*fact(n-1)
	movl	%eax, -4(%ebp) ;向地址%ebp-4中传入%eax
.L1:
	movl	-4(%ebp), %eax ;向%eax中传入地址%ebp-4中的数据,对应返回值
	leave
	ret                    ;函数返回值
\end{lstlisting}

\section{Exercise 2}
针对以下C程序,给出标号\texttt{L}处变量\texttt{j}可能的值集合。
\begin{lstlisting}[language = C++, 
         numbers=left, 
         numberstyle=\tiny,
         keywordstyle=\bfseries\color{blue!70},
         commentstyle=\color{red!40!green!60!blue},
         frame=shadowbox,
         rulesepcolor=\color{red!20!green!30!blue!20},
         basicstyle=\ttfamily]
int main()
{
    int i,j = 0;
    for(i = 0; i < 10; i++)
    {
        switch(i)
        {
            case 0:case 2: break;
            case 3:case 5: continue;
            default: j = i;  
        }
     L: j += i * 2;
    }
}
\end{lstlisting}

\paragraph{解}
对每一次循环迭代进行分析:\par
$\texttt{i}=0$时:\texttt{break}退出\texttt{switch}语句,执行\texttt{L: j += i * 2},$\texttt{j}=0$; \par
$\texttt{i}=1$时:执行\texttt{default: j = i},再执行\texttt{L: j += i * 2},$\texttt{j}=3$; \par
$\texttt{i}=2$时:\texttt{break}退出\texttt{switch}语句,执行\texttt{L: j += i * 2},$\texttt{j}=7$; \par
$\texttt{i}=3$时:\texttt{continue}进入下一次循环; \par
$\texttt{i}=4$时:执行\texttt{default: j = i},再执行\texttt{L: j += i * 2},$\texttt{j}=12$; \par
$\texttt{i}=5$时:\texttt{continue}进入下一次循环; \par
$\texttt{i} \geq 6$时:执行\texttt{default: j = i},再执行\texttt{L: j += i * 2},$\texttt{j}=3\texttt{i}=18,\,21,\,24,\,27$; \par
$\texttt{i} \geq 10$时:循环结束。 \par
综上,\text{j}可能的值集合为$\{0,3,7,12,18,21,24,27\}$。
\\

\section{Exercise 3}
针对以下C/C++程序:\par
(1)补全相关代码\par
(2)用文字简要描述变量b和p的类型信息。\par
如变量\texttt{a}的类型信息描述如下:变量\text{a}是一个含10个元素的数组,每个元素是指向一个整型变量的指针。
\begin{lstlisting}[language = C++, 
         numbers=left, 
         numberstyle=\tiny,
         keywordstyle=\bfseries\color{blue!70},
         commentstyle=\color{red!40!green!60!blue},
         frame=shadowbox,
         rulesepcolor=\color{red!20!green!30!blue!20},
         basicstyle=\ttfamily,
         escapeinside=``]
int main()
{
    int  i;
    int* q;
    int* a[10];
    int* (*b[10])[10];
    int* (*(*p)[10])[10];
      
    i = 100; q = &i; a[1] = q; b[1] = &a; p = &b;

    cout <<`\underline{~~~~~\texttt{p[0]}~~~~~}`<< endl; //输出100,待补全

    cout <<`\underline{~~~~~~\texttt{*p}~~~~~~}`<< endl; //输出100,待补全

}
\end{lstlisting}

\paragraph{解}
\subparagraph{(1)}
补全如下:
\begin{lstlisting}[language = C++, 
         keywordstyle=\bfseries\color{blue!70},
         commentstyle=\color{red!40!green!60!blue},
         frame=shadowbox,
         rulesepcolor=\color{red!20!green!30!blue!20},
         basicstyle=\ttfamily,
         escapeinside=``]
    cout <<`\underline{\texttt{{\color{red!20!green!60!blue}{*(*}}p[0]{\color{red!20!green!60!blue}{[1])[1]}}}}`<< endl; //输出100

    cout <<`\underline{\texttt{{\color{red!20!green!60!blue}{**(*(}}*p{\color{red!20!green!60!blue}{+1)+1)}}}}`<< endl; //输出100
\end{lstlisting}
\subparagraph{(2)}
变量\texttt{b}是一个含10个元素的数组,每个元素是一个指针,指向一个含10个元素的数组,被指向数组的每个元素是一个指向整型变量的指针;或者可以说变量\texttt{b}是一个$10\times10$的二维数组,每个元素是一个指向整型变量的指针。\par
变量\texttt{p}是一个指针,指向一个含10个元素的数组,数组的每个元素是一个指向含10个元素数组的指针,被指向数组的每个元素是一个指向整型变量的指针;或者可以说变量\texttt{p}是一个指向$10\times10$二维数组的指针,数组的每个元素是一个指向整型变量的指针。

\end{document}
